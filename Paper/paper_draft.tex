\documentclass[12pt]{article}
\usepackage[margin=1in]{geometry}
\usepackage{setspace}
\usepackage{graphicx}
\usepackage{booktabs}
\usepackage{amsmath}
\usepackage{natbib}
\usepackage{hyperref}
\usepackage{threeparttable}
\usepackage{float}

\doublespacing

\title{Can Fintech Fill the Gap? Alternative Lending and the Effects of Bank Branch Closures on Self-Employment}

\author{Alina Malkova\thanks{Department of Economics, Florida Institute of Technology. Email: amalkova@fit.edu. I thank [acknowledgments].}}

\date{\today}

\begin{document}

\maketitle

\begin{abstract}
Bank branch closures disproportionately affect low-to-moderate income communities, reducing access to credit for small business formation. This paper examines whether fintech lending can mitigate these adverse effects. Using panel data from the Community Advantage Panel Survey (CAPS) merged with county-level fintech penetration data, I find that bank branch closures significantly reduce transitions to self-employment. In the baseline specification, counties with higher fintech market share show attenuated closure effects. However, the results are sensitive to identification strategy: when controlling for county fixed effects or county $\times$ year fixed effects, the interaction coefficient shrinks and loses significance. A placebo test finds no pre-trend, providing some reassurance. The most cautious interpretation is that fintech may partially mitigate closure effects, but the magnitude is uncertain due to potential confounding from time-varying county characteristics. These findings highlight both the promise and the identification challenges in studying fintech's role in financial inclusion.

\medskip
\noindent\textbf{Keywords:} Fintech, Bank Branch Closures, Self-Employment, Financial Inclusion, Alternative Credit

\medskip
\noindent\textbf{JEL Codes:} G21, G23, L26, R12
\end{abstract}

\newpage

\section{Introduction}

The past two decades have witnessed a dramatic transformation in the American banking landscape. Between 2009 and 2019, more than 13,000 bank branches closed across the United States, with closures disproportionately concentrated in low-income and minority communities \citep{nguyen2019}. These closures have significant consequences for local economic activity, as bank branches provide critical services beyond deposit-taking---including relationship lending, credit assessment, and financial advice that are particularly valuable for small business formation \citep{petersen2002}.

Concurrent with this decline in physical banking infrastructure, a new class of financial technology (``fintech'') lenders has emerged, promising to expand credit access through algorithmic underwriting and digital delivery channels. Fintech lenders claim to reach borrowers underserved by traditional banks, using alternative data sources and machine learning to assess creditworthiness beyond traditional credit scores \citep{jagtiani2019}. Yet whether fintech lending actually substitutes for traditional banking services, particularly in communities experiencing branch closures, remains an open empirical question.

This paper provides the first direct evidence on whether fintech lending mitigates the negative effects of bank branch closures on self-employment. I combine individual-level panel data from the Community Advantage Panel Survey (CAPS) with county-level measures of fintech mortgage market penetration from \citet{fuster2019}. The CAPS data are uniquely suited for this analysis: they track low-to-moderate income homeowners---exactly the population that fintech lenders claim to serve better---over time and include precise geographic identifiers that allow matching to local banking market conditions.

My identification strategy builds on \citet{malkova2024}, who show that bank branch closures induced by merger-related consolidation reduce transitions to incorporated self-employment among affected individuals. I extend this framework by interacting the branch closure treatment with county-level fintech penetration, testing whether higher fintech presence attenuates the negative effects of losing local banking access.

The main findings can be summarized as follows. First, in the 2010-2014 period when fintech data are available, bank branch closures significantly reduce self-employment transitions. A closure in an individual's ZIP code reduces the probability of transitioning to incorporated self-employment by approximately 1.2 percentage points (p $<$ 0.05). Notably, this negative effect is specific to the fintech sample period; in 2003-2009, the closure coefficient is positive (though insignificant), highlighting important temporal heterogeneity.

Second, the baseline negative effect appears attenuated in counties with higher fintech penetration. The interaction between closures and fintech market share is positive ($\beta$ = 0.82, p $<$ 0.05). At mean fintech penetration (4.0\%), the net closure effect is approximately -0.9 percentage points. At fintech penetration of 5.1\%, the negative effect is fully offset.

However, these results are fragile. When I add county fixed effects to control for time-invariant county characteristics, the interaction coefficient drops from 0.82 to 0.28 and loses significance. In a horse-race specification including fintech, broadband, and social capital interactions simultaneously, none of the interactions achieves significance. The small sample size (N = 484, representing only 147 individuals across 59 counties) limits statistical power and generalizability.

These findings contribute to several literatures. First, I add to the growing body of work on fintech lending and financial inclusion \citep{buchak2018, fuster2019, jagtiani2019}. While prior studies document that fintech lenders serve different borrower populations than traditional banks, I investigate the potential substitutability of these lending channels in a setting where traditional access is exogenously reduced.

Second, I contribute to research on bank branch closures and local economic development \citep{nguyen2019, grennan2020}. The baseline results suggest fintech presence may partially attenuate closure effects, though identification challenges prevent strong causal claims.

Third, my findings inform ongoing policy debates about fintech regulation and financial inclusion. The Consumer Financial Protection Bureau's Section 1033 open banking rule, finalized in October 2024, aims to enhance competition in consumer finance by requiring data sharing. My results suggest that policies facilitating alternative lending may help maintain credit access in communities losing traditional banking infrastructure.

The remainder of this paper proceeds as follows. Section 2 reviews related literature and develops hypotheses. Section 3 describes the data sources and construction of key variables. Section 4 presents the empirical methodology. Section 5 reports main results and robustness checks. Section 6 discusses implications and concludes.


\section{Background and Hypotheses}

\subsection{Bank Branch Closures and Credit Access}

Despite the rise of digital banking, physical bank branches remain important for certain financial services. \citet{petersen2002} document that distance to lender matters for small business credit access, and while technology has reduced this friction over time, relationship lending still depends on proximity. Bank branches serve as information production centers where loan officers can observe ``soft information'' about borrowers that is difficult to transmit through digital channels \citep{stein2002}.

Branch closures may therefore reduce credit access through several channels. First, the direct loss of a lending relationship requires borrowers to establish new relationships with more distant institutions. Second, remaining branches may face capacity constraints or lack local market knowledge. Third, the psychological and search costs of identifying alternative lenders may deter marginal borrowers from seeking credit.

Recent empirical work confirms these concerns. \citet{nguyen2019} find that branch closures reduce small business lending and local employment, with effects concentrated in areas with fewer remaining branches. \citet{malkova2024} shows that merger-induced closures reduce transitions to incorporated self-employment, consistent with credit constraints affecting business formation decisions.

\subsection{Fintech Lending as a Substitute}

Fintech lenders have grown rapidly since the 2008 financial crisis, expanding from a negligible market share to originating over 10\% of mortgages by 2017 \citep{buchak2018}. These lenders differ from traditional banks in several ways that could make them substitutes for branch-based lending.

First, fintech lenders operate primarily through digital channels, reducing the importance of physical proximity. Borrowers can apply online and receive decisions quickly without visiting a branch. \citet{fuster2019} show that fintech lenders process applications faster and with less sensitivity to local competition.

Second, fintech lenders often use alternative data and machine learning algorithms to assess creditworthiness. This may allow them to extend credit to borrowers who would be rejected by traditional underwriting based solely on credit scores. \citet{jagtiani2019} find evidence that fintech lenders serve borrowers in ZIP codes with lower credit scores and higher credit constraints.

Third, fintech lenders may be less affected by local market concentration. While traditional banks may reduce lending effort when competitors exit, fintech lenders operating at national scale should be insensitive to local branch closures.

However, there are also reasons why fintech might not substitute for traditional banking. Fintech lenders primarily operate in mortgage markets and may not offer the full range of small business credit products. The relationship lending and advisory services provided by bank branches may not have digital equivalents. And concerns about algorithmic bias suggest that fintech may replicate or exacerbate existing disparities in credit access \citep{bartlett2022}.

\subsection{Empirical Questions}

Based on this discussion, I investigate the following empirical questions:

\textbf{Question 1:} Do bank branch closures reduce transitions to self-employment, consistent with credit constraints affecting business formation?

\textbf{Question 2:} Is the negative effect of branch closures on self-employment attenuated in counties with higher fintech penetration?

\textbf{Question 3:} If so, is this mitigating effect specific to fintech lending rather than general digital infrastructure or social capital?

I frame these as questions rather than hypotheses because the identification strategy has important limitations (discussed in Section 5.5) that prevent definitive causal claims.


\section{Data}

\subsection{Community Advantage Panel Survey (CAPS)}

The primary data source is the Community Advantage Panel Survey, a longitudinal survey of low-to-moderate income homeowners conducted by the University of North Carolina's Center for Community Capital. CAPS respondents were drawn from participants in the Community Advantage Program, an affordable lending initiative, and have been surveyed annually since 2003.

CAPS is uniquely suited for studying fintech and credit access for several reasons. First, the sample consists of low-to-moderate income borrowers---exactly the population that fintech lenders claim to serve better and that faces the greatest constraints when traditional banking access is reduced. Second, the panel structure allows me to track the same individuals over time and control for individual fixed effects. Third, CAPS includes precise geographic identifiers (ZIP codes) that allow matching to local banking market conditions.

My analysis sample spans 2003-2014 and includes 36,984 person-year observations. The main outcome variables are indicators for transitioning to incorporated self-employment (\textit{anytoise}) and unincorporated self-employment (\textit{anytouse}) in the following year.

Table \ref{tab:summary} presents summary statistics. Approximately 1.8\% of the sample transitions to incorporated self-employment in a given year, while 2.3\% transition to unincorporated self-employment. The mean branch closure measure is -0.009, indicating slight net increases in nearby branches on average, though 10\% of the sample experienced closures in their ZIP code.

\subsection{Bank Branch Closures}

Branch closure data come from the FDIC Summary of Deposits, which provides annual information on all bank branch locations. Following \citet{malkova2024}, I construct a measure of bank branch closures at the ZIP code level that captures changes in local banking access.

The closure variable (\textit{closure\_zip}) measures the change in bank branches in the respondent's ZIP code. Negative values indicate net closures, while positive values indicate net openings. The treatment indicator (\textit{treat\_zip}) equals one if the individual lives in a ZIP code that experienced any closure during the sample period.

\subsection{Fintech Penetration}

County-level fintech mortgage market share comes from \citet{fuster2019}, who classify mortgage lenders in Home Mortgage Disclosure Act (HMDA) data as fintech based on their lending technology. Fintech lenders are defined as those that primarily originate loans through online channels and use automated underwriting.

I merge this data to CAPS using ZIP-to-county crosswalks from the Census Bureau. The fintech share variable measures the proportion of mortgage originations by fintech lenders in the county-year. Importantly, CAPS data extend only through 2014, so the effective analysis period is 2010-2014 (not 2010-2017 as in the full Fuster et al. data).

Mean fintech share in the sample is 4.0\%, with a standard deviation of 2.4\%. Fintech penetration varies substantially across counties, from less than 0.2\% to over 25\% in some areas.

\textbf{Measurement limitation}: The fintech variable captures \textit{mortgage} market share, not small business lending. The implicit mechanism connecting mortgage fintech to self-employment is that fintech-enabled home equity access may fund business formation. Home equity is a primary source of startup capital for small businesses, particularly for low-to-moderate income entrepreneurs who lack access to traditional business credit. However, this link is indirect, and I cannot directly test whether individuals use mortgage refinancing or home equity for business purposes. Ideally, one would measure fintech small business lending directly (e.g., from platforms like OnDeck, Kabbage, or LendingClub's small business products), but such data are not available at the county-year level for this period.

\subsection{Geographic Controls}

I supplement the main data with several geographic variables that may confound or moderate the fintech-closure relationship:

\textit{Banking Access}: County-level branch density (branches per 10,000 population) from FDIC Summary of Deposits. I also construct a banking desert indicator for counties with fewer than 1 branch per 10,000 residents.

\textit{Social Capital}: Economic connectedness measures from the Social Capital Atlas \citep{chetty2022}, which captures cross-class social connections using Facebook friendship data.

\textit{Digital Infrastructure}: Broadband access rates from the American Community Survey, measuring the percentage of households with broadband internet subscriptions.


\section{Empirical Methodology}

\subsection{Baseline Specification}

I estimate the effect of branch closures on self-employment using the following specification:

\begin{equation}
Y_{it} = \beta_1 \text{Closure}_{zt} + \mathbf{X}_{it}'\gamma + \alpha_m + \delta_t + \varepsilon_{it}
\end{equation}

where $Y_{it}$ is an indicator for transitioning to self-employment, $\text{Closure}_{zt}$ measures bank branch closures in the individual's ZIP code, $\mathbf{X}_{it}$ is a vector of time-varying controls, $\alpha_m$ are merger-group fixed effects, and $\delta_t$ are year fixed effects. Standard errors are clustered at the county level.

\textbf{Important note on fixed effects structure}: Following \citet{malkova2024}, identification comes from merger-group (mergerID) fixed effects rather than individual fixed effects. This approach exploits variation within groups of individuals affected by the same bank merger event. Individuals are assigned to merger groups based on their exposure to specific merger-induced branch closures. The merger-group fixed effects absorb time-invariant characteristics common to individuals exposed to the same merger, while allowing comparison of closure effects across differently affected individuals within each merger group.

This specification has implications for sample size. The CAPS panel contains 36,984 person-year observations, but the reghdfe estimator drops singleton observations (individuals observed only once within their merger group). Combined with the 2010-2014 restriction for fintech analysis, this reduces the effective sample to 484 observations representing 147 individuals across 10 merger groups and 59 counties.

\subsection{Fintech Interaction Specification}

To investigate whether fintech mitigates closure effects, I estimate:

\begin{equation}
Y_{it} = \beta_1 \text{Closure}_{zt} + \beta_2 \text{Fintech}_{ct} + \beta_3 (\text{Closure}_{zt} \times \text{Fintech}_{ct}) + \alpha_m + \delta_t + \varepsilon_{it}
\end{equation}

The coefficient $\beta_3$ captures the differential effect of closures in high- versus low-fintech areas. Under Question 2, if fintech mitigates closure effects, we expect $\beta_1 < 0$ (closures reduce self-employment) and $\beta_3 > 0$ (fintech attenuates this effect).

The identifying assumption is that, conditional on merger-group and year fixed effects, fintech penetration is not correlated with unobserved time-varying factors that also affect the closure-self-employment relationship. This assumption is strong: fintech penetration reflects county-level economic conditions, demographics, and technology adoption that may independently affect self-employment resilience. Robustness checks with county and county $\times$ year fixed effects (Section 5.5) help assess sensitivity to this assumption.

\subsection{Alternative Mechanisms}

To distinguish fintech-specific effects from general digital connectivity, I estimate additional specifications interacting closures with broadband access and social capital:

\begin{equation}
Y_{it} = \beta_1 \text{Closure}_{zt} + \beta_2 Z_{ct} + \beta_3 (\text{Closure}_{zt} \times Z_{ct}) + \alpha_m + \delta_t + \varepsilon_{it}
\end{equation}

where $Z_{ct}$ is broadband access or economic connectedness. If fintech effects operate through general digital infrastructure, these interactions should also be significant.


\section{Results}

\subsection{Main Results: Banking Access and Self-Employment}

Table \ref{tab:table2} presents baseline results for incorporated self-employment. Column 1 shows that branch closures have a positive but insignificant effect in the full sample (coefficient = 0.005, p = 0.36). This differs from the significant negative effect found in \citet{malkova2024}, potentially due to the smaller estimation sample when restricting to observations with all geographic controls.

Columns 2-3 add branch density and its interaction with closures. The interaction term is positive but not significant, suggesting that the level of remaining banking infrastructure does not significantly moderate closure effects in this sample.

\subsection{Fintech Penetration and Self-Employment}

Table \ref{tab:table3} presents the key results for the fintech interaction, restricting to 2010-2014 when both CAPS and fintech data are available.

\textbf{Important caveat on sign flip}: The closure coefficient is positive (though insignificant) in Table 2 but negative (significant) in Table 3. This sign change is \textit{not} driven by adding fintech controls---it reflects the different time periods. Running the baseline specification separately by period (without fintech) shows:
\begin{itemize}
    \item 2003-2009: Closure coefficient = +0.016 (SE = 0.011, p = 0.17)
    \item 2010-2014: Closure coefficient = -0.012 (SE = 0.006, p = 0.04)
\end{itemize}
The negative closure effect is specific to the 2010-2014 period. This could reflect differential macroeconomic conditions (the financial crisis recovery), changes in the nature of closures, or compositional shifts in the sample. This temporal heterogeneity complicates interpretation of the fintech interaction, which is identified only in the period where closures already have negative effects.

Column 1 shows that in this period, branch closures significantly reduce incorporated self-employment (coefficient = -0.013, p = 0.026). The fintech share coefficient is negative but not significant.

Column 2 adds the closure-fintech interaction. The main closure effect becomes larger in magnitude (-0.042, p = 0.013), while the interaction term is positive and significant (0.82, p = 0.027). This pattern is consistent with Hypothesis 2: closures reduce self-employment, but this effect is attenuated in high-fintech areas.

To interpret the magnitudes: at mean fintech share (4.0\%), the net closure effect is $-0.042 + 0.82 \times 0.040 = -0.009$, or about 0.9 percentage points. The fintech level that fully offsets the closure effect is $0.042 / 0.82 = 0.051$, or 5.1\%---slightly above the mean but well within the observed range.

Column 3 adds branch density as a control, and the results are similar. However, as discussed in Section 5.5, these results are sensitive to specification and should be interpreted cautiously.

\textbf{Sample size concern}: The fintech analysis sample contains only 484 observations (after dropping singletons in the fixed effects estimation), representing 147 unique individuals across 59 counties in 10 merger groups. This small sample limits statistical power and raises concerns about generalizability. The dramatic reduction from the full CAPS sample (N = 36,984) reflects both the time restriction (fintech data available only 2010-2014) and the demanding fixed effects structure.

\subsection{Alternative Mechanisms}

Table \ref{tab:table4} tests whether the mitigating effect is specific to fintech or reflects general digital connectivity.

Column 1 shows that economic connectedness (social capital) does not significantly moderate closure effects. The interaction coefficient is positive but small and insignificant.

Column 2 examines broadband access. Again, the interaction is not significant, and the point estimate is actually negative (though tiny).

Column 3 includes both variables simultaneously with similar null results.

The pattern is consistent with Question 3---broadband and social capital do not moderate closure effects---but should be interpreted cautiously given the overall fragility of the fintech interaction itself. In a horse-race specification including fintech, broadband, and social capital interactions simultaneously (reported in Appendix Table A3), none of the interactions achieves statistical significance, including fintech. The null results for broadband and social capital are therefore consistent with, but do not strongly support, a fintech-specific channel.

\subsection{Unincorporated Self-Employment}

Table \ref{tab:table5} examines unincorporated self-employment. Column 1 shows that closures significantly reduce unincorporated self-employment in the full sample (coefficient = -0.033, p = 0.032).

Column 2 presents the fintech interaction for unincorporated self-employment. Unlike the positive interaction found for incorporated self-employment, the fintech interaction here is negative and insignificant (-0.10, SE = 0.68). This suggests that fintech does not mitigate closure effects for unincorporated businesses. The contrast between incorporated and unincorporated self-employment may reflect different credit needs: incorporated businesses may rely more on formal credit markets where fintech operates, while unincorporated businesses may depend more on informal networks and local relationships that fintech cannot substitute.

The branch density and social capital interactions (Columns 3-4) are not statistically significant for unincorporated self-employment either.


\subsection{Robustness and Identification}

A key concern with the main results is that fintech penetration is not randomly assigned. Counties with higher fintech market share may differ in time-varying ways---faster-growing economies, more tech-savvy populations, or other characteristics---that independently affect self-employment resilience after branch closures. Merger-group and year fixed effects do not address county-level time-varying confounders.

Table \ref{tab:robust} presents robustness checks addressing this concern. Column 1 reproduces the baseline specification. Column 2 adds county fixed effects, controlling for time-invariant county heterogeneity. The closure-fintech interaction coefficient drops from 0.82 to 0.28 and loses statistical significance. Column 3 includes county $\times$ year fixed effects, the most demanding specification that absorbs all county-level time variation. The interaction coefficient is 0.46 but remains insignificant.

These results suggest caution in interpreting the main findings. The baseline specification may overstate the mitigating effect of fintech if high-fintech counties also experienced other favorable trends during 2010-2014.

\subsection{Placebo Test: Pre-Fintech Period}

A placebo test provides some reassurance about the baseline results. I assign each observation its county's \textit{future} fintech penetration (2010-2012 average) and test whether this predicts differential closure effects in the \textit{pre-fintech} period (2003-2009). If the main result reflected pre-existing county differences rather than fintech itself, we would expect areas that later adopted fintech to already show attenuated closure effects before fintech existed.

\begin{table}[H]
\centering
\caption{Placebo Test: Future Fintech and Pre-Period Closure Effects}
\label{tab:placebo}
\begin{threeparttable}
\begin{tabular}{lcc}
\toprule
 & (1) & (2) \\
 & 2003-2009 & 2010-2014 \\
 & (Placebo) & (Main) \\
\midrule
Closure (ZIP) & 0.0336 & -0.0420** \\
 & (0.0307) & (0.0165) \\
Future Fintech Share & 0.5164 & --- \\
 & (0.5931) &  \\
Closure $\times$ Future Fintech & -0.7194 & --- \\
 & (0.6220) &  \\
Fintech Share & --- & -0.8286 \\
 &  & (0.8620) \\
Closure $\times$ Fintech & --- & 0.8224** \\
 &  & (0.3617) \\
\midrule
Observations & 963 & 484 \\
MergerID FE & Yes & Yes \\
Year FE & Yes & Yes \\
\bottomrule
\end{tabular}
\begin{tablenotes}
\small
\item \textit{Notes:} Column 1 tests whether future fintech penetration (2010-2012 average) predicts differential closure effects in the pre-fintech period (2003-2009). A significant positive interaction in column 1 would suggest pre-existing differences confound the main result. The null result provides reassurance. Standard errors clustered by county. * p<0.10, ** p<0.05, *** p<0.01.
\end{tablenotes}
\end{threeparttable}
\end{table}

The placebo interaction coefficient is -0.72 (p = 0.25), suggesting no pre-trend. Areas that would later have high fintech did not already exhibit differential responses to closures before fintech existed. This is the strongest piece of evidence supporting a causal interpretation of the main results.

\subsection{Summary of Evidence}

The pattern of results---significant in the baseline, attenuated but positive with county FE, and no pre-trend in the placebo---is consistent with fintech providing some mitigation, though the magnitude is uncertain. The most conservative interpretation is that the true effect lies between zero and the baseline estimate of 0.82, with considerable uncertainty.


\begin{table}[H]
\centering
\caption{Identification Robustness}
\label{tab:robust}
\begin{threeparttable}
\begin{tabular}{lccc}
\toprule
 & (1) & (2) & (3) \\
 & Baseline & +County FE & County$\times$Year FE \\
\midrule
Closure (ZIP) & -0.0420** & -0.0178 & -0.0198 \\
 & (0.0165) & (0.0244) & (0.0310) \\
Fintech Share & -0.8286 & -0.8239 & --- \\
 & (0.8620) & (0.9902) &  \\
Closure $\times$ Fintech & 0.8224** & 0.2787 & 0.4591 \\
 & (0.3617) & (0.5537) & (0.7717) \\
\midrule
Observations & 484 & 474 & 379 \\
MergerID FE & Yes & Yes & Yes \\
Year FE & Yes & Yes & Absorbed \\
County FE & No & Yes & Absorbed \\
County $\times$ Year FE & No & No & Yes \\
\bottomrule
\end{tabular}
\begin{tablenotes}
\small
\item \textit{Notes:} Sample restricted to 2010-2014 (overlap of CAPS and fintech data). Dependent variable is transition to incorporated self-employment. MergerID and year FE included in all columns. Standard errors clustered by county. Fintech share main effect absorbed by county $\times$ year FE in column 3. * p<0.10, ** p<0.05, *** p<0.01.
\end{tablenotes}
\end{threeparttable}
\end{table}


\section{Discussion and Conclusion}

This paper investigates whether fintech lending mitigates the negative effects of bank branch closures on self-employment. The baseline results show a significant positive interaction: closures reduce incorporated self-employment by 4.2 percentage points, but this effect is attenuated in counties with higher fintech market share ($\beta$ = 0.82, p $<$ 0.05). However, the effect is not found for unincorporated self-employment, where the fintech interaction is null.

The identification strategy has important limitations that prevent strong causal claims. Fintech penetration is not randomly assigned, and counties with higher fintech may differ in time-varying ways that independently affect self-employment resilience. When I add county fixed effects or county $\times$ year fixed effects, the interaction coefficient shrinks substantially (to 0.28 and 0.46, respectively) and loses statistical significance. In a horse-race specification with broadband and social capital interactions, none of the interaction terms achieves significance. This fragility suggests the baseline may overstate the true effect.

The placebo test provides the strongest evidence for a causal interpretation: areas that would later have high fintech showed no differential response to closures in 2003-2009, before fintech existed. The null result on pre-trends is reassuring, though not definitive.

The most honest interpretation is that the true mitigating effect likely lies between zero and the baseline estimate, with considerable uncertainty. The evidence is consistent with fintech providing some benefit for incorporated self-employment, but is not conclusive.

Additional caveats apply. The analysis is limited to 2010-2014 when CAPS and fintech data overlap. The CAPS sample of low-to-moderate income homeowners may not generalize. I observe fintech mortgage market share but cannot measure fintech small business lending directly. Sample sizes are small (N = 484 in the fintech analysis), limiting power for demanding specifications.

For policy, these findings suggest fintech \textit{may} partially offset banking consolidation effects, but should not substitute for direct interventions to maintain credit access. Even under the most optimistic interpretation, fintech does not fully mitigate closure harms, and the evidence for any mitigation is uncertain.

Future research should address identification more directly through instrumental variables (using pre-period broadband infrastructure or distance to fintech headquarters), event studies exploiting staggered fintech market entry, or direct measurement of fintech small business lending. Understanding whether and how fintech expands credit access remains an important open question.


\newpage
\begin{table}[H]
\centering
\caption{Summary Statistics}
\label{tab:summary}
\begin{threeparttable}
\begin{tabular}{lcccc}
\toprule
Variable & Mean & Std. Dev. & Min & Max \\
\midrule
\multicolumn{5}{l}{\textit{Panel A: Main Variables}} \\
Transition to Incorporated SE & 0.018 & 0.132 & 0 & 1 \\
Transition to Unincorporated SE & 0.023 & 0.150 & 0 & 1 \\
Branch Closure (ZIP) & -0.009 & 0.242 & -4 & 4.78 \\
Treatment Indicator & 0.101 & 0.302 & 0 & 1 \\
\midrule
\multicolumn{5}{l}{\textit{Panel B: Geographic Alternative Data}} \\
Fintech Share & 0.040 & 0.024 & 0.002 & 0.254 \\
Branches per 10,000 & 2.317 & 0.865 & 0 & 17.62 \\
Banking Desert & 0.015 & 0.123 & 0 & 1 \\
Economic Connectedness & 0.753 & 0.197 & 0.30 & 1.57 \\
Broadband Access (\%) & 80.07 & 9.20 & 0 & 100 \\
\bottomrule
\end{tabular}
\begin{tablenotes}
\small
\item \textit{Notes:} Summary statistics from CAPS panel data, 2003-2014. N = 36,984 person-years. Fintech share available only for 2010-2014 (N = 13,244 in full sample; N = 484 after merger-group FE estimation).
\end{tablenotes}
\end{threeparttable}
\end{table}


\begin{table}[H]
\centering
\caption{Banking Access and Incorporated Self-Employment}
\label{tab:table2}
\begin{threeparttable}
\begin{tabular}{lccc}
\toprule
 & (1) & (2) & (3) \\
 & Baseline & +Branches & +Interaction \\
\midrule
Closure (ZIP) & 0.0053 & 0.0051 & -0.0019 \\
 & (0.0057) & (0.0057) & (0.0114) \\
Branches per 10k &  & -0.0091 & -0.0091 \\
 &  & (0.0140) & (0.0140) \\
Closure $\times$ Branches &  &  & 0.0032 \\
 &  &  & (0.0053) \\
\midrule
Observations & 1,449 & 1,449 & 1,449 \\
MergerID FE & Yes & Yes & Yes \\
Year FE & Yes & Yes & Yes \\
\bottomrule
\end{tabular}
\begin{tablenotes}
\small
\item \textit{Notes:} Dependent variable is indicator for transitioning to incorporated self-employment. MergerID and year fixed effects included. Standard errors clustered by county in parentheses. Banking desert specification omitted due to collinearity with merger-group FE (insufficient within-group variation). * p<0.10, ** p<0.05, *** p<0.01.
\end{tablenotes}
\end{threeparttable}
\end{table}


\begin{table}[H]
\centering
\caption{Fintech Penetration and Self-Employment (2010-2014)}
\label{tab:table3}
\begin{threeparttable}
\begin{tabular}{lccc}
\toprule
 & (1) & (2) & (3) \\
 & Fintech & +Interaction & +Branches \\
\midrule
Closure (ZIP) & -0.0130** & -0.0420** & -0.0419** \\
 & (0.0057) & (0.0165) & (0.0168) \\
Fintech Share & -0.8327 & -0.8286 & -0.8253 \\
 & (0.8617) & (0.8620) & (0.8621) \\
Closure $\times$ Fintech &  & 0.8224** & 0.8174** \\
 &  & (0.3617) & (0.3655) \\
Branches per 10k &  &  & 0.0015 \\
 &  &  & (0.0208) \\
\midrule
Observations & 484 & 484 & 484 \\
MergerID FE & Yes & Yes & Yes \\
Year FE & Yes & Yes & Yes \\
\bottomrule
\end{tabular}
\begin{tablenotes}
\small
\item \textit{Notes:} Sample restricted to 2010-2014 (overlap of CAPS and fintech data). Dependent variable is indicator for transitioning to incorporated self-employment. MergerID and year fixed effects included. Standard errors clustered by county in parentheses. * p<0.10, ** p<0.05, *** p<0.01.
\end{tablenotes}
\end{threeparttable}
\end{table}


\begin{table}[H]
\centering
\caption{Social Capital and Digital Access}
\label{tab:table4}
\begin{threeparttable}
\begin{tabular}{lccc}
\toprule
 & (1) & (2) & (3) \\
 & Econ Connect & Broadband & Combined \\
\midrule
Closure (ZIP) & -0.0036 & 0.0173 & 0.0431 \\
 & (0.0152) & (0.0280) & (0.0397) \\
Economic Connectedness & 0.0742 &  & 0.0756 \\
 & (0.0658) &  & (0.0816) \\
Closure $\times$ Econ Connect & 0.0117 &  & 0.0354 \\
 & (0.0224) &  & (0.0447) \\
Broadband Access (\%) &  & 0.0005 & -0.0000 \\
 &  & (0.0008) & (0.0008) \\
Closure $\times$ Broadband &  & -0.0001 & -0.0008 \\
 &  & (0.0003) & (0.0008) \\
\midrule
Observations & 1,418 & 1,447 & 1,418 \\
MergerID FE & Yes & Yes & Yes \\
Year FE & Yes & Yes & Yes \\
\bottomrule
\end{tabular}
\begin{tablenotes}
\small
\item \textit{Notes:} Dependent variable is indicator for transitioning to incorporated self-employment. MergerID and year fixed effects included. Standard errors clustered by county in parentheses. * p<0.10, ** p<0.05, *** p<0.01.
\end{tablenotes}
\end{threeparttable}
\end{table}


\begin{table}[H]
\centering
\caption{Unincorporated Self-Employment}
\label{tab:table5}
\begin{threeparttable}
\begin{tabular}{lcccc}
\toprule
 & (1) & (2) & (3) & (4) \\
 & Baseline & Fintech & Branches & Social Capital \\
\midrule
Closure (ZIP) & -0.0331** & -0.0293 & -0.0863 & 0.0155 \\
 & (0.0151) & (0.0308) & (0.0693) & (0.0554) \\
Fintech Share &  & -0.7284 &  &  \\
 &  & (1.1923) &  &  \\
Closure $\times$ Fintech &  & -0.1011 &  &  \\
 &  & (0.6842) &  &  \\
Branches per 10k &  &  & -0.0014 &  \\
 &  &  & (0.0066) &  \\
Closure $\times$ Branches &  &  & 0.0245 &  \\
 &  &  & (0.0288) &  \\
Economic Connectedness &  &  &  & 0.0438* \\
 &  &  &  & (0.0251) \\
Closure $\times$ Econ Connect &  &  &  & -0.0616 \\
 &  &  &  & (0.0773) \\
\midrule
Observations & 1,449 & 484 & 1,449 & 1,417 \\
MergerID FE & Yes & Yes & Yes & Yes \\
Year FE & Yes & Yes & Yes & Yes \\
\bottomrule
\end{tabular}
\begin{tablenotes}
\small
\item \textit{Notes:} Dependent variable is indicator for transitioning to unincorporated self-employment. Column 2 restricted to 2010-2014 (fintech data period). Standard errors clustered by county in parentheses. * p<0.10, ** p<0.05, *** p<0.01.
\end{tablenotes}
\end{threeparttable}
\end{table}


\newpage
\bibliographystyle{aer}
\bibliography{references}

\end{document}
