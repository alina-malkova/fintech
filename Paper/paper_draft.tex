\documentclass[12pt]{article}
\usepackage[margin=1in]{geometry}
\usepackage{setspace}
\usepackage{graphicx}
\usepackage{booktabs}
\usepackage{amsmath}
\usepackage{natbib}
\usepackage{hyperref}
\usepackage{threeparttable}
\usepackage{float}

\doublespacing

\title{Can Fintech Fill the Gap? Alternative Lending and the Effects of Bank Branch Closures on Self-Employment}

\author{Alina Malkova\thanks{Department of Economics, Florida Institute of Technology. Email: amalkova@fit.edu. I thank [acknowledgments].}}

\date{\today}

\begin{document}

\maketitle

\begin{abstract}
Bank branch closures disproportionately affect low-to-moderate income communities, reducing access to credit for small business formation. This paper examines whether fintech lending can mitigate these adverse effects and whether fintech-style creditworthiness assessment can identify underserved borrowers. Using panel data from the Community Advantage Panel Survey (CAPS) merged with county-level fintech penetration data, I find that bank branch closures significantly reduce transitions to self-employment. In the baseline specification, counties with higher fintech market share show attenuated closure effects, though results are sensitive to identification strategy. As a complementary analysis, I construct a ``fintech-style'' creditworthiness score from CAPS behavioral variables---payment history, income stability, financial resilience---and validate it against mortgage default. The score achieves exceptional predictive power (AUC = 0.926): respondents in the bottom tercile have a 52\% default rate versus 0\% in the top tercile. Financial resilience (emergency fund access, home equity) is the key discriminator. Individuals with higher resilience scores show smaller negative effects from branch closures (p = 0.098), suggesting that the characteristics fintech lenders value provide some buffer against credit supply shocks. However, linking CAPS to HMDA mortgage origination data reveals that ZIP-level fintech market presence does \textit{not} buffer closure effects---if anything, areas with higher fintech activity experience slightly larger negative effects ($\beta$ = 0.0048, p = 0.054). This suggests fintech lenders enter markets where banks retreat but do not substitute for relationship lending that supports small business formation. These findings highlight both the promise of alternative credit assessment and the identification challenges in studying fintech's role in financial inclusion.

\medskip
\noindent\textbf{Keywords:} Fintech, Bank Branch Closures, Self-Employment, Financial Inclusion, Alternative Credit, Creditworthiness

\medskip
\noindent\textbf{JEL Codes:} G21, G23, L26, R12
\end{abstract}

\newpage

\section{Introduction}

The past two decades have witnessed a dramatic transformation in the American banking landscape. Between 2009 and 2019, more than 13,000 bank branches closed across the United States, with closures disproportionately concentrated in low-income and minority communities \citep{nguyen2019}. These closures have significant consequences for local economic activity, as bank branches provide critical services beyond deposit-taking---including relationship lending, credit assessment, and financial advice that are particularly valuable for small business formation \citep{petersen2002}.

Concurrent with this decline in physical banking infrastructure, a new class of financial technology (``fintech'') lenders has emerged, promising to expand credit access through algorithmic underwriting and digital delivery channels. Fintech lenders claim to reach borrowers underserved by traditional banks, using alternative data sources and machine learning to assess creditworthiness beyond traditional credit scores \citep{jagtiani2019}. Yet whether fintech lending actually substitutes for traditional banking services, particularly in communities experiencing branch closures, remains an open empirical question.

This paper provides the first direct evidence on whether fintech lending mitigates the negative effects of bank branch closures on self-employment. I combine individual-level panel data from the Community Advantage Panel Survey (CAPS) with county-level measures of fintech mortgage market penetration from \citet{fuster2019}. The CAPS data are uniquely suited for this analysis: they track low-to-moderate income homeowners---exactly the population that fintech lenders claim to serve better---over time and include precise geographic identifiers that allow matching to local banking market conditions.

My identification strategy builds on \citet{malkova2024}, who show that bank branch closures induced by merger-related consolidation reduce transitions to incorporated self-employment among affected individuals. I extend this framework by interacting the branch closure treatment with county-level fintech penetration, testing whether higher fintech presence attenuates the negative effects of losing local banking access.

The main findings can be summarized as follows. First, in the 2010-2014 period when fintech data are available, bank branch closures significantly reduce self-employment transitions. A closure in an individual's ZIP code reduces the probability of transitioning to incorporated self-employment by approximately 1.2 percentage points (p $<$ 0.05). Notably, this negative effect is specific to the fintech sample period; in 2003-2009, the closure coefficient is positive (though insignificant), highlighting important temporal heterogeneity.

Second, the baseline negative effect appears attenuated in counties with higher fintech penetration. The interaction between closures and fintech market share is positive ($\beta$ = 0.82, p $<$ 0.05). At mean fintech penetration (4.0\%), the net closure effect is approximately -0.9 percentage points. At fintech penetration of 5.1\%, the negative effect is fully offset.

However, these results are fragile. When I add county fixed effects to control for time-invariant county characteristics, the interaction coefficient drops from 0.82 to 0.28 and loses significance. In a horse-race specification including fintech, broadband, and social capital interactions simultaneously, none of the interactions achieves significance. The small sample size (N = 484, representing only 147 individuals across 59 counties) limits statistical power and generalizability.

These findings contribute to several literatures. First, I add to the growing body of work on fintech lending and financial inclusion \citep{buchak2018, fuster2019, jagtiani2019}. While prior studies document that fintech lenders serve different borrower populations than traditional banks, I investigate the potential substitutability of these lending channels in a setting where traditional access is exogenously reduced.

Second, I contribute to research on bank branch closures and local economic development \citep{nguyen2019, grennan2020}. The baseline results suggest fintech presence may partially attenuate closure effects, though identification challenges prevent strong causal claims.

Third, I develop and validate a ``fintech-style'' creditworthiness methodology using individual-level behavioral variables from CAPS. The resulting score achieves exceptional predictive power for mortgage default (AUC = 0.926), demonstrating that the alternative data sources fintech lenders claim to use are indeed predictive of loan performance. This individual-level analysis complements the county-level fintech penetration results and does not depend on fintech adoption being exogenous.

Fourth, my findings inform ongoing policy debates about fintech regulation and financial inclusion. The Consumer Financial Protection Bureau's Section 1033 open banking rule, finalized in October 2024, aims to enhance competition in consumer finance by requiring data sharing. My results suggest that policies facilitating alternative lending may help maintain credit access in communities losing traditional banking infrastructure.

The remainder of this paper proceeds as follows. Section 2 reviews related literature and develops the empirical questions. Section 3 describes the data sources and construction of key variables. Section 4 presents the empirical methodology. Section 5 reports main results and robustness checks. Section 6 develops the alternative creditworthiness analysis using individual-level behavioral variables, then links CAPS respondents to HMDA mortgage origination data to test whether fintech lender presence directly buffers closure effects. Section 7 discusses implications and concludes.


\section{Background and Hypotheses}

\subsection{Bank Branch Closures and Credit Access}

Despite the rise of digital banking, physical bank branches remain important for certain financial services. \citet{petersen2002} document that distance to lender matters for small business credit access, and while technology has reduced this friction over time, relationship lending still depends on proximity. Bank branches serve as information production centers where loan officers can observe ``soft information'' about borrowers that is difficult to transmit through digital channels \citep{stein2002}.

Branch closures may therefore reduce credit access through several channels. First, the direct loss of a lending relationship requires borrowers to establish new relationships with more distant institutions. Second, remaining branches may face capacity constraints or lack local market knowledge. Third, the psychological and search costs of identifying alternative lenders may deter marginal borrowers from seeking credit.

Recent empirical work confirms these concerns. \citet{nguyen2019} find that branch closures reduce small business lending and local employment, with effects concentrated in areas with fewer remaining branches. \citet{malkova2024} shows that merger-induced closures reduce transitions to incorporated self-employment, consistent with credit constraints affecting business formation decisions.

\subsection{Fintech Lending as a Substitute}

Fintech lenders have grown rapidly since the 2008 financial crisis, expanding from a negligible market share to originating over 10\% of mortgages by 2017 \citep{buchak2018}. These lenders differ from traditional banks in several ways that could make them substitutes for branch-based lending.

First, fintech lenders operate primarily through digital channels, reducing the importance of physical proximity. Borrowers can apply online and receive decisions quickly without visiting a branch. \citet{fuster2019} show that fintech lenders process applications faster and with less sensitivity to local competition.

Second, fintech lenders often use alternative data and machine learning algorithms to assess creditworthiness. This may allow them to extend credit to borrowers who would be rejected by traditional underwriting based solely on credit scores. \citet{jagtiani2019} find evidence that fintech lenders serve borrowers in ZIP codes with lower credit scores and higher credit constraints.

Third, fintech lenders may be less affected by local market concentration. While traditional banks may reduce lending effort when competitors exit, fintech lenders operating at national scale should be insensitive to local branch closures.

However, there are also reasons why fintech might not substitute for traditional banking. Fintech lenders primarily operate in mortgage markets and may not offer the full range of small business credit products. The relationship lending and advisory services provided by bank branches may not have digital equivalents. And concerns about algorithmic bias suggest that fintech may replicate or exacerbate existing disparities in credit access \citep{bartlett2022}.

\subsection{Empirical Questions}

Based on this discussion, I investigate the following empirical questions:

\textbf{Question 1:} Do bank branch closures reduce transitions to self-employment, consistent with credit constraints affecting business formation?

\textbf{Question 2:} Is the negative effect of branch closures on self-employment attenuated in counties with higher fintech penetration?

\textbf{Question 3:} If so, is this mitigating effect specific to fintech lending rather than general digital infrastructure or social capital?

I frame these as questions rather than hypotheses because the identification strategy has important limitations (discussed in Section 5.5) that prevent definitive causal claims.


\section{Data}

\subsection{Community Advantage Panel Survey (CAPS)}

The primary data source is the Community Advantage Panel Survey, a longitudinal survey of low-to-moderate income homeowners conducted by the University of North Carolina's Center for Community Capital. CAPS respondents were drawn from participants in the Community Advantage Program, an affordable lending initiative, and have been surveyed annually since 2003.

CAPS is uniquely suited for studying fintech and credit access for several reasons. First, the sample consists of low-to-moderate income borrowers---exactly the population that fintech lenders claim to serve better and that faces the greatest constraints when traditional banking access is reduced. Second, the panel structure allows me to track the same individuals over time and control for merger-group-level unobservables. Third, CAPS includes precise geographic identifiers (ZIP codes) that allow matching to local banking market conditions.

My analysis sample spans 2003-2014 and includes 36,984 person-year observations. The main outcome variables are indicators for transitioning to incorporated self-employment (\textit{anytoise}) and unincorporated self-employment (\textit{anytouse}) in the following year.

Table \ref{tab:summary} presents summary statistics. Approximately 1.8\% of the sample transitions to incorporated self-employment in a given year, while 2.3\% transition to unincorporated self-employment. The mean branch closure measure is -0.009, indicating slight net increases in nearby branches on average, though 10\% of the sample experienced closures in their ZIP code.

\subsection{Bank Branch Closures}

Branch closure data come from the FDIC Summary of Deposits, which provides annual information on all bank branch locations. Following \citet{malkova2024}, I construct a measure of bank branch closures at the ZIP code level that captures changes in local banking access.

The closure variable (\textit{closure\_zip}) measures the change in bank branches in the respondent's ZIP code. Negative values indicate net closures, while positive values indicate net openings. The treatment indicator (\textit{treat\_zip}) equals one if the individual lives in a ZIP code that experienced any closure during the sample period.

\subsection{Fintech Penetration}

County-level fintech mortgage market share comes from \citet{fuster2019}, who classify mortgage lenders in Home Mortgage Disclosure Act (HMDA) data as fintech based on their lending technology. Fintech lenders are defined as those that primarily originate loans through online channels and use automated underwriting.

I merge this data to CAPS using ZIP-to-county crosswalks from the Census Bureau. The fintech share variable measures the proportion of mortgage originations by fintech lenders in the county-year. Importantly, CAPS data extend only through 2014, so the effective analysis period is 2010-2014 (not 2010-2017 as in the full Fuster et al. data).

Mean fintech share in the sample is 4.0\%, with a standard deviation of 2.4\%. Fintech penetration varies substantially across counties, from less than 0.2\% to over 25\% in some areas.

\textbf{Measurement limitation}: The fintech variable captures \textit{mortgage} market share, not small business lending. The implicit mechanism connecting mortgage fintech to self-employment is that fintech-enabled home equity access may fund business formation. Home equity is a primary source of startup capital for small businesses, particularly for low-to-moderate income entrepreneurs who lack access to traditional business credit. However, this link is indirect, and I cannot directly test whether individuals use mortgage refinancing or home equity for business purposes. Ideally, one would measure fintech small business lending directly (e.g., from platforms like OnDeck, Kabbage, or LendingClub's small business products), but such data are not available at the county-year level for this period.

\subsection{Geographic Controls}

I supplement the main data with several geographic variables that may confound or moderate the fintech-closure relationship:

\textit{Banking Access}: County-level branch density (branches per 10,000 population) from FDIC Summary of Deposits. I also construct a banking desert indicator for counties with fewer than 1 branch per 10,000 residents.

\textit{Social Capital}: Economic connectedness measures from the Social Capital Atlas \citep{chetty2022}, which captures cross-class social connections using Facebook friendship data.

\textit{Digital Infrastructure}: Broadband access rates from the American Community Survey, measuring the percentage of households with broadband internet subscriptions.


\section{Empirical Methodology}

\subsection{Baseline Specification}

I estimate the effect of branch closures on self-employment using the following specification:

\begin{equation}
Y_{it} = \beta_1 \text{Closure}_{zt} + \mathbf{X}_{it}'\gamma + \alpha_m + \delta_t + \varepsilon_{it}
\end{equation}

where $Y_{it}$ is an indicator for transitioning to self-employment, $\text{Closure}_{zt}$ measures bank branch closures in the individual's ZIP code, $\mathbf{X}_{it}$ is a vector of time-varying controls, $\alpha_m$ are merger-group fixed effects, and $\delta_t$ are year fixed effects. Standard errors are clustered at the county level.

\textbf{Important note on fixed effects structure}: Following \citet{malkova2024}, identification comes from merger-group (mergerID) fixed effects rather than individual fixed effects. This approach exploits variation within groups of individuals affected by the same bank merger event. Individuals are assigned to merger groups based on their exposure to specific merger-induced branch closures. The merger-group fixed effects absorb time-invariant characteristics common to individuals exposed to the same merger, while allowing comparison of closure effects across differently affected individuals within each merger group.

This specification has implications for sample size. The CAPS panel contains 36,984 person-year observations, but the reghdfe estimator drops singleton observations (individuals observed only once within their merger group). Combined with the 2010-2014 restriction for fintech analysis, this reduces the effective sample to 484 observations representing 147 individuals across 10 merger groups and 59 counties.

\subsection{Fintech Interaction Specification}

To investigate whether fintech mitigates closure effects, I estimate:

\begin{equation}
Y_{it} = \beta_1 \text{Closure}_{zt} + \beta_2 \text{Fintech}_{ct} + \beta_3 (\text{Closure}_{zt} \times \text{Fintech}_{ct}) + \alpha_m + \delta_t + \varepsilon_{it}
\end{equation}

The coefficient $\beta_3$ captures the differential effect of closures in high- versus low-fintech areas. Under Question 2, if fintech mitigates closure effects, we expect $\beta_1 < 0$ (closures reduce self-employment) and $\beta_3 > 0$ (fintech attenuates this effect).

The identifying assumption is that, conditional on merger-group and year fixed effects, fintech penetration is not correlated with unobserved time-varying factors that also affect the closure-self-employment relationship. This assumption is strong: fintech penetration reflects county-level economic conditions, demographics, and technology adoption that may independently affect self-employment resilience. Robustness checks with county and county $\times$ year fixed effects (Section 5.5) help assess sensitivity to this assumption.

\subsection{Alternative Mechanisms}

To distinguish fintech-specific effects from general digital connectivity, I estimate additional specifications interacting closures with broadband access and social capital:

\begin{equation}
Y_{it} = \beta_1 \text{Closure}_{zt} + \beta_2 Z_{ct} + \beta_3 (\text{Closure}_{zt} \times Z_{ct}) + \alpha_m + \delta_t + \varepsilon_{it}
\end{equation}

where $Z_{ct}$ is broadband access or economic connectedness. If fintech effects operate through general digital infrastructure, these interactions should also be significant.


\section{Results}

\subsection{Main Results: Banking Access and Self-Employment}

Table \ref{tab:table2} presents baseline results for incorporated self-employment. Column 1 shows that branch closures have a positive but insignificant effect in the full sample (coefficient = 0.005, p = 0.36). This differs from the significant negative effect found in \citet{malkova2024}, potentially due to the smaller estimation sample when restricting to observations with all geographic controls.

Columns 2-3 add branch density and its interaction with closures. The interaction term is positive but not significant, suggesting that the level of remaining banking infrastructure does not significantly moderate closure effects in this sample.

\subsection{Fintech Penetration and Self-Employment}

Table \ref{tab:table3} presents the key results for the fintech interaction, restricting to 2010-2014 when both CAPS and fintech data are available.

\textbf{Important caveat on sign flip}: The closure coefficient is positive (though insignificant) in Table 2 but negative (significant) in Table 3. This sign change is \textit{not} driven by adding fintech controls---it reflects the different time periods. Running the baseline specification separately by period (without fintech) shows:
\begin{itemize}
    \item 2003-2009: Closure coefficient = +0.016 (SE = 0.011, p = 0.17)
    \item 2010-2014: Closure coefficient = -0.012 (SE = 0.006, p = 0.04)
\end{itemize}
The negative closure effect is specific to the 2010-2014 period. This could reflect differential macroeconomic conditions (the financial crisis recovery), changes in the nature of closures, or compositional shifts in the sample. This temporal heterogeneity complicates interpretation of the fintech interaction, which is identified only in the period where closures already have negative effects.

Column 1 shows that in this period, branch closures significantly reduce incorporated self-employment (coefficient = -0.013, p = 0.026). The fintech share coefficient is negative but not significant.

Column 2 adds the closure-fintech interaction. The main closure effect becomes larger in magnitude (-0.042, p = 0.013), while the interaction term is positive and significant (0.82, p = 0.027). This pattern is consistent with Question 2: closures reduce self-employment, but this effect is attenuated in high-fintech areas.

To interpret the magnitudes: at mean fintech share (4.0\%), the net closure effect is $-0.042 + 0.82 \times 0.040 = -0.009$, or about 0.9 percentage points. The fintech level that fully offsets the closure effect is $0.042 / 0.82 = 0.051$, or 5.1\%---slightly above the mean but well within the observed range.

\textbf{Note on the fintech main effect}: The coefficient on fintech share itself is consistently large and negative (around -0.83), though imprecise (SE = 0.86). While not statistically significant, this pattern is worth noting. It could suggest that in areas \textit{without} branch closures, fintech may compete with or displace traditional lending in ways that do not benefit self-employment---or it may simply reflect omitted variable bias from unobserved county characteristics correlated with fintech adoption. The positive interaction coefficient dominates in closure-affected areas, but the negative main effect hints at more nuanced effects of fintech competition that merit future investigation.

Column 3 adds branch density as a control, and the results are similar. However, as discussed in Section 5.5, these results are sensitive to specification and should be interpreted cautiously.

\textbf{Sample size concern}: The fintech analysis sample contains only 484 observations (after dropping singletons in the fixed effects estimation), representing 147 unique individuals across 59 counties in 10 merger groups. This small sample limits statistical power and raises concerns about generalizability. The dramatic reduction from the full CAPS sample (N = 36,984) reflects both the time restriction (fintech data available only 2010-2014) and the demanding fixed effects structure.

\subsection{Alternative Mechanisms}

Table \ref{tab:table4} tests whether the mitigating effect is specific to fintech or reflects general digital connectivity.

Column 1 shows that economic connectedness (social capital) does not significantly moderate closure effects. The interaction coefficient is positive but small and insignificant.

Column 2 examines broadband access. Again, the interaction is not significant, and the point estimate is actually negative (though tiny).

Column 3 includes both variables simultaneously with similar null results.

The pattern is consistent with Question 3---broadband and social capital do not moderate closure effects---but should be interpreted cautiously given the overall fragility of the fintech interaction itself. In a horse-race specification including fintech, broadband, and social capital interactions simultaneously (reported in Appendix Table A3), none of the interactions achieves statistical significance, including fintech. The null results for broadband and social capital are therefore consistent with, but do not strongly support, a fintech-specific channel.

\subsection{Unincorporated Self-Employment}

Table \ref{tab:table5} examines unincorporated self-employment. Column 1 shows that closures significantly reduce unincorporated self-employment in the full sample (coefficient = -0.033, p = 0.032).

Column 2 presents the fintech interaction for unincorporated self-employment. Unlike the positive interaction found for incorporated self-employment, the fintech interaction here is negative and insignificant (-0.10, SE = 0.68). This null result provides a within-paper falsification test that does not depend on the identification concerns affecting the main analysis. If fintech mortgage share were simply proxying for ``economically dynamic county'' or other time-varying confounders, we would expect it to attenuate closure effects for \textit{both} types of self-employment. The fact that the positive interaction is specific to incorporated self-employment---which is more dependent on formal credit markets---supports the lending channel interpretation.

Incorporated businesses typically require more startup capital and are more likely to seek formal financing, making them more sensitive to changes in credit access. Unincorporated self-employment (e.g., freelancing, gig work, informal businesses) may depend more on informal networks and personal savings, channels that fintech mortgage lending cannot substitute. This differential response is exactly what we would predict if the mechanism operates through formal credit access rather than general county characteristics.

The branch density and social capital interactions (Columns 3-4) are not statistically significant for unincorporated self-employment either.


\subsection{Robustness and Identification}

A key concern with the main results is that fintech penetration is not randomly assigned. Counties with higher fintech market share may differ in time-varying ways---faster-growing economies, more tech-savvy populations, or other characteristics---that independently affect self-employment resilience after branch closures. Merger-group and year fixed effects do not address county-level time-varying confounders.

Table \ref{tab:robust} presents robustness checks addressing this concern. Column 1 reproduces the baseline specification. Column 2 adds county fixed effects, controlling for time-invariant county heterogeneity. The closure-fintech interaction coefficient drops from 0.82 to 0.28 and loses statistical significance. Column 3 includes county $\times$ year fixed effects, the most demanding specification that absorbs all county-level time variation. The interaction coefficient is 0.46 but remains insignificant.

These results suggest caution in interpreting the main findings. However, the pattern is not unambiguously negative. The coefficient in the county $\times$ year specification (0.46) is roughly half the baseline estimate rather than collapsing to zero, and the sample drops to only 379 observations---asking a lot of such limited data. The maintained positive sign, while statistically inconclusive, is at least not contradictory to the baseline finding.

A power calculation clarifies the interpretation. To detect an effect of 0.82 (the baseline estimate) at 80\% power with $\alpha = 0.05$ requires a standard error below 0.29.\footnote{Minimum detectable effect = $(z_{0.975} + z_{0.80}) \times SE = 2.8 \times SE$, so detecting $\beta = 0.82$ requires $SE < 0.82/2.8 = 0.29$.} The county $\times$ year specification has $SE = 0.77$, implying only about 18\% power to detect the baseline effect. Even the county FE specification ($SE = 0.55$) has roughly 32\% power. Both specifications are severely underpowered, so the loss of significance reflects ``we cannot tell'' rather than ``the effect is not there.'' The minimum detectable effect at 80\% power in the county $\times$ year specification is 2.16---more than 2.5 times the baseline estimate---highlighting how demanding this test is given the sample size.

\textbf{Closure intensity}: An additional test examines whether fintech mitigation is stronger when closures are more severe. Replacing the continuous closure measure with a ``severe closure'' indicator (losing more than half of local branches) yields a closure $\times$ fintech interaction of 1.12 (SE = 0.50), larger than the baseline estimate. This is consistent with fintech filling credit gaps that become more acute when closures are more binding.

\textbf{Sample composition}: The sign flip in the closure main effect---positive in 2003--2009 (+0.016), negative in 2010--2014 (-0.012)---could reflect either genuine time variation (post-crisis credit conditions) or compositional changes from panel attrition. Restricting to a balanced panel of individuals observed in both periods, the sign flip persists and is even slightly larger (+0.019 in pre-period, -0.022 in post-period). This confirms the sign flip reflects genuine time variation in how closures affect self-employment, not selection from attrition.

\subsection{Placebo Test: Pre-Fintech Period}

A placebo test provides some reassurance about the baseline results. I assign each observation its county's \textit{future} fintech penetration (2010-2012 average) and test whether this predicts differential closure effects in the \textit{pre-fintech} period (2003-2009). If the main result reflected pre-existing county differences rather than fintech itself, we would expect areas that later adopted fintech to already show attenuated closure effects before fintech existed.

\begin{table}[H]
\centering
\caption{Placebo Test: Future Fintech and Pre-Period Closure Effects}
\label{tab:placebo}
\begin{threeparttable}
\begin{tabular}{lcc}
\toprule
 & (1) & (2) \\
 & 2003-2009 & 2010-2014 \\
 & (Placebo) & (Main) \\
\midrule
Closure (ZIP) & 0.0336 & -0.0420** \\
 & (0.0307) & (0.0165) \\
Future Fintech Share & 0.5164 & --- \\
 & (0.5931) &  \\
Closure $\times$ Future Fintech & -0.7194 & --- \\
 & (0.6220) &  \\
Fintech Share & --- & -0.8286 \\
 &  & (0.8620) \\
Closure $\times$ Fintech & --- & 0.8224** \\
 &  & (0.3617) \\
\midrule
Observations & 963 & 484 \\
MergerID FE & Yes & Yes \\
Year FE & Yes & Yes \\
\bottomrule
\end{tabular}
\begin{tablenotes}
\small
\item \textit{Notes:} Column 1 tests whether future fintech penetration (2010-2012 average) predicts differential closure effects in the pre-fintech period (2003-2009). A significant positive interaction in column 1 would suggest pre-existing differences confound the main result. The null result provides reassurance. Standard errors clustered by county. * p<0.10, ** p<0.05, *** p<0.01.
\end{tablenotes}
\end{threeparttable}
\end{table}

The placebo interaction coefficient is -0.72 (p = 0.25), which is not only insignificant but \textit{opposite in sign} from the main result (+0.82). This sign reversal is informative: if time-invariant county characteristics (e.g., economic dynamism, tech-savviness) explained the positive fintech interaction in 2010-2014, we would expect the same counties to show attenuated closure effects in 2003-2009 as well. Instead, the point estimate is negative, actively cutting against the confounding story. Areas that would later have high fintech did not already exhibit differential responses to closures before fintech existed. This is the strongest piece of evidence supporting a causal interpretation of the main results.

\subsection{Leave-One-Out Analysis by Merger Group}

With only 10 merger groups driving identification (8 of which contribute to the fintech analysis after singleton drops), it is important to verify that no single merger group drives the entire result. Table \ref{tab:loo} presents a leave-one-out analysis, re-estimating the fintech interaction after dropping each merger group in turn.

\begin{table}[H]
\centering
\caption{Leave-One-Out Analysis by Merger Group}
\label{tab:loo}
\begin{threeparttable}
\begin{tabular}{lccc}
\toprule
Dropped Merger Group & Coefficient & Std. Error & N \\
\midrule
None (Baseline) & 0.8224** & (0.3617) & 484 \\
\midrule
Drop Merger 2 & 0.9620** & (0.3838) & 462 \\
Drop Merger 3 & 0.8587** & (0.3775) & 477 \\
Drop Merger 5 & 0.8713 & (0.6456) & 206 \\
Drop Merger 6 & 0.8384** & (0.3634) & 471 \\
Drop Merger 10 & 0.7768** & (0.3729) & 459 \\
Drop Merger 11 & 0.9365* & (0.5608) & 388 \\
Drop Merger 12 & 0.9297*** & (0.3436) & 456 \\
Drop Merger 13 & 0.5596* & (0.3233) & 469 \\
\midrule
\multicolumn{4}{l}{\textit{Range: 0.56--0.96 (8 contributing merger groups)}} \\
\bottomrule
\end{tabular}
\begin{tablenotes}
\small
\item \textit{Notes:} Each row drops one merger group and re-estimates the baseline specification. Merger groups 1 and 7 are omitted because they contribute zero observations to the fintech sample after the 2010--2014 restriction and singleton drops. MergerID and year FE included. Standard errors clustered by county. * p<0.10, ** p<0.05, *** p<0.01.
\end{tablenotes}
\end{threeparttable}
\end{table}

All leave-one-out coefficients remain in the range 0.56--0.96, confirming that the result is robust to the exclusion of any single merger group. No single merger drives the finding: the lowest coefficient (0.56 when dropping Merger 13) remains positive and economically meaningful, while the highest (0.96 when dropping Merger 2) stays within a plausible range. This provides confidence that the fintech mitigation effect is not an artifact of one influential observation.

\subsection{County-Level Triangulation}

A concern with the CAPS analysis is that it relies on only 147 individuals across 59 counties. To address external validity, I conduct a complementary county-level analysis using County Business Patterns data from the Census Bureau, which covers over 3,000 counties annually from 2010--2017.

Table \ref{tab:county} reports results from regressing county-level establishment growth on measures of banking access interacted with fintech penetration. The outcome is the annual percentage change in total business establishments. I use two measures of limited banking access: (1) a ``banking desert'' indicator for counties with fewer than 1 branch per 10,000 population, and (2) a ``low branch'' indicator for counties below median branch density.

\begin{table}[H]
\centering
\caption{County-Level Triangulation: Establishment Growth}
\label{tab:county}
\begin{threeparttable}
\begin{tabular}{lccc}
\toprule
 & (1) & (2) & (3) \\
 & Desert $\times$ Fintech & +County FE & Low Branch $\times$ Fintech \\
\midrule
Banking Desert & -0.116 & --- & \\
 & (0.327) & & \\
Low Branch & & & 0.175 \\
 & & & (0.146) \\
Fintech Share & -3.459*** & -0.432 & -4.169*** \\
 & (0.763) & (1.352) & (0.807) \\
Desert $\times$ Fintech & 8.793** & -4.798 & \\
 & (4.298) & (12.252) & \\
Low Branch $\times$ Fintech & & & 6.731*** \\
 & & & (2.091) \\
\midrule
Observations & 21,023 & 21,003 & 21,027 \\
Counties & 3,120 & 3,100 & 3,121 \\
Year FE & Yes & Yes & Yes \\
County FE & No & Yes & No \\
\bottomrule
\end{tabular}
\begin{tablenotes}
\small
\item \textit{Notes:} Dependent variable is annual establishment growth rate (\%). Banking Desert = county with $<$1 branch per 10,000 population. Low Branch = county below median branch density. Standard errors clustered by county. * p<0.10, ** p<0.05, *** p<0.01.
\end{tablenotes}
\end{threeparttable}
\end{table}

The county-level results mirror the individual-level CAPS findings. In the baseline specification with year fixed effects, the banking desert $\times$ fintech interaction is positive and significant ($\beta$ = 8.79, p $<$ 0.05), indicating that banking-underserved counties with higher fintech penetration experienced stronger establishment growth. The low branch $\times$ fintech interaction is even more precisely estimated ($\beta$ = 6.73, p $<$ 0.01). As with the CAPS analysis, adding county fixed effects absorbs the cross-sectional variation and the interaction loses significance.

The consistency across a sample of 3,120 counties provides important triangulation. While the county-level identification is weaker (no merger-group variation creating quasi-exogenous closure timing), the fact that both analyses show the same pattern---significant interaction in cross-sectional specifications, attenuation with more demanding fixed effects---strengthens confidence that the CAPS findings are not idiosyncratic to 147 individuals.

\subsection{Summary of Evidence}

The pattern of results---significant in the baseline, attenuated but positive with county FE, no pre-trend in the placebo, robust to leave-one-out analysis (coefficients range 0.56--0.96), and replicated in county-level data with over 3,000 counties---is consistent with fintech providing some mitigation, though the magnitude is uncertain. The most conservative interpretation is that the true effect lies between zero and the baseline estimate of 0.82, with considerable uncertainty.


\begin{table}[H]
\centering
\caption{Identification Robustness}
\label{tab:robust}
\begin{threeparttable}
\begin{tabular}{lccc}
\toprule
 & (1) & (2) & (3) \\
 & Baseline & +County FE & County$\times$Year FE \\
\midrule
Closure (ZIP) & -0.0420** & -0.0178 & -0.0198 \\
 & (0.0165) & (0.0244) & (0.0310) \\
Fintech Share & -0.8286 & -0.8239 & --- \\
 & (0.8620) & (0.9902) &  \\
Closure $\times$ Fintech & 0.8224** & 0.2787 & 0.4591 \\
 & (0.3617) & (0.5537) & (0.7717) \\
\midrule
Observations & 484 & 474 & 379 \\
MergerID FE & Yes & Yes & Yes \\
Year FE & Yes & Yes & Absorbed \\
County FE & No & Yes & Absorbed \\
County $\times$ Year FE & No & No & Yes \\
\bottomrule
\end{tabular}
\begin{tablenotes}
\small
\item \textit{Notes:} Sample restricted to 2010-2014 (overlap of CAPS and fintech data). Dependent variable is transition to incorporated self-employment. MergerID and year FE included in all columns. Standard errors clustered by county. Fintech share main effect absorbed by county $\times$ year FE in column 3. * p<0.10, ** p<0.05, *** p<0.01.
\end{tablenotes}
\end{threeparttable}
\end{table}


\subsection{Graphical Evidence}

Figure \ref{fig:spec_curve} presents a specification curve analysis showing the robustness of the fintech mitigation effect across different model specifications. The coefficients are ordered by magnitude, with the baseline specification (MergerID + Year FE, county clustering) highlighted. The key finding is that positive and significant effects emerge only with merger-group fixed effects---specifications using county or individual fixed effects yield null results. This pattern underscores that identification relies on within-merger-group variation in fintech exposure.

\begin{figure}[H]
\centering
\includegraphics[width=0.95\textwidth]{../Output/specification_curve_clean.png}
\caption{Specification Curve: Robustness of Fintech Mitigation Effect}
\label{fig:spec_curve}
\begin{minipage}{0.9\textwidth}
\footnotesize
\textit{Notes:} Each point represents the coefficient on Closure $\times$ Fintech from a different specification. Error bars show 95\% confidence intervals. The baseline specification (red diamond) uses merger-group and year fixed effects with county-clustered standard errors. Specifications are ordered by coefficient magnitude. The pattern shows that positive effects require merger-group fixed effects; county FE and individual FE specifications yield null results.
\end{minipage}
\end{figure}

Figure \ref{fig:event_study} presents an event study examining self-employment dynamics around branch closures. The coefficients at $t = -4, -3, -2$ (pre-closure) are close to zero and statistically insignificant, supporting the parallel trends assumption. Post-closure coefficients show small negative effects, though confidence intervals are wide due to limited sample size.

\begin{figure}[H]
\centering
\includegraphics[width=0.95\textwidth]{../Output/event_study_clean.png}
\caption{Event Study: Self-Employment Dynamics Around Branch Closures}
\label{fig:event_study}
\begin{minipage}{0.9\textwidth}
\footnotesize
\textit{Notes:} Coefficients represent the effect on self-employment transition probability at each year relative to branch closure (year $t = 0$). The reference period is $t = -1$. Shaded area shows 95\% confidence intervals. Pre-closure coefficients near zero support the parallel trends assumption. Individual and year fixed effects included; standard errors clustered by county.
\end{minipage}
\end{figure}

Figure \ref{fig:coef_compare} compares key specifications to illustrate why the identification strategy matters. The fintech mitigation effect is statistically significant only in the baseline specification with merger-group fixed effects. The post-2012 subsample shows a larger (marginally significant) effect, consistent with fintech becoming more prevalent and effective over time.

\begin{figure}[H]
\centering
\includegraphics[width=0.85\textwidth]{../Output/coefficient_comparison.png}
\caption{Key Specification Comparison: Fintech Mitigation Effect by Model}
\label{fig:coef_compare}
\begin{minipage}{0.9\textwidth}
\footnotesize
\textit{Notes:} Bar heights represent the coefficient on Closure $\times$ Fintech from each specification. Error bars show 95\% confidence intervals. Green = significant at 5\%; orange = significant at 10\%; gray = not significant. The baseline MergerID FE specification shows a significant positive interaction ($\beta = 0.82$, $p < 0.05$), while specifications without merger-group fixed effects yield null results.
\end{minipage}
\end{figure}

\subsection{Digital Infrastructure as a Mechanism}

If fintech lending genuinely substitutes for traditional banking, the effect should be concentrated in areas with adequate digital infrastructure. Fintech lenders operate primarily through online channels, so their services are inaccessible to populations without reliable internet. I test this mechanism by estimating a triple interaction: Closure $\times$ Fintech $\times$ Broadband.

Table \ref{tab:triple} presents results. Column 1 reproduces the baseline Closure $\times$ Fintech interaction. Column 2 adds Closure $\times$ Broadband. Column 3 estimates the full triple interaction.

\begin{table}[H]
\centering
\caption{Triple Interaction: Closure $\times$ Fintech $\times$ Broadband}
\label{tab:triple}
\begin{threeparttable}
\begin{tabular}{lccc}
\toprule
 & (1) & (2) & (3) \\
 & Baseline & +Broadband & Triple \\
\midrule
Closure $\times$ Fintech & 0.0244** & 0.0260** & 0.0385* \\
 & (0.0107) & (0.0110) & (0.0214) \\
Closure $\times$ Broadband &  & -0.0121 & -0.0141 \\
 &  & (0.0123) & (0.0089) \\
Fintech $\times$ Broadband &  &  & 0.0304** \\
 &  &  & (0.0151) \\
Closure $\times$ Fintech $\times$ Broadband &  &  & -0.0192 \\
 &  &  & (0.0256) \\
\midrule
Observations & 484 & 484 & 484 \\
Merger-Year FE & Yes & Yes & Yes \\
\bottomrule
\end{tabular}
\begin{tablenotes}
\small
\item \textit{Notes:} Dependent variable is transition to incorporated self-employment. Fintech share and broadband access standardized (mean 0, SD 1). Broadband = percent of households with broadband internet (ACS 2019). Standard errors clustered by county in parentheses. * p<0.10, ** p<0.05, *** p<0.01.
\end{tablenotes}
\end{threeparttable}
\end{table}

The triple interaction itself is not statistically significant ($\beta$ = -0.019, t = -0.75), but a split-sample analysis is more revealing. In high-broadband areas (above-median internet penetration), the Closure $\times$ Fintech interaction is positive and marginally significant ($\beta$ = 0.032, t = 1.84, p = 0.066). In low-broadband areas, the interaction is near zero and insignificant ($\beta$ = -0.012, t = -0.40). This pattern is consistent with fintech requiring digital infrastructure to function as a substitute for physical banking.

The Fintech $\times$ Broadband interaction in Column 3 is positive and significant ($\beta$ = 0.030, t = 2.01), indicating that fintech and broadband are complements---areas with both high fintech penetration and high broadband access show better self-employment outcomes. This finding supports the hypothesis that digital infrastructure enables fintech to serve as an alternative to traditional banking.

\subsection{Temporal Dynamics}

If fintech genuinely substitutes for traditional banking, the mitigation effect should strengthen as the fintech industry matures and gains scale. I test this hypothesis by splitting the sample into early (2010-2012) and late (2013-2014) periods and estimating the Closure $\times$ Fintech interaction separately.

\begin{table}[H]
\centering
\caption{Temporal Dynamics: Fintech Effects by Period}
\label{tab:temporal}
\begin{threeparttable}
\begin{tabular}{lcc}
\toprule
 & (1) & (2) \\
 & Early (2010-2012) & Late (2013-2014) \\
\midrule
Closure (ZIP) & -0.0089 & -0.0156** \\
 & (0.0105) & (0.0072) \\
Closure $\times$ Fintech & 0.0183 & 0.0290*** \\
 & (0.0190) & (0.0093) \\
\midrule
t-statistic & 0.96 & 3.12 \\
p-value & 0.337 & 0.002 \\
Observations & 289 & 195 \\
\bottomrule
\end{tabular}
\begin{tablenotes}
\small
\item \textit{Notes:} Dependent variable is transition to incorporated self-employment. Fintech share standardized (mean 0, SD 1). Merger-year fixed effects included; standard errors clustered by county. * p<0.10, ** p<0.05, *** p<0.01.
\end{tablenotes}
\end{threeparttable}
\end{table}

Table \ref{tab:temporal} reveals a striking pattern. In the early period (2010-2012), when fintech lenders were still nascent, the Closure $\times$ Fintech interaction is positive but not statistically significant ($\beta$ = 0.018, t = 0.96). In the late period (2013-2014), as fintech gained market share and operational scale, the interaction is larger and highly significant ($\beta$ = 0.029, t = 3.12, p = 0.002).

This temporal pattern is consistent with learning and scale effects. Early fintech lenders may have lacked the infrastructure, underwriting refinement, and brand recognition to effectively substitute for traditional banking relationships. By 2013-2014, the industry had matured---Quicken Loans and other online lenders had captured substantial market share, and borrowers had become more comfortable with digital lending channels. The strengthening effect over time provides additional support for the hypothesis that fintech can mitigate branch closure effects, conditional on sufficient industry maturity.


\section{Alternative Creditworthiness Analysis}

The preceding analysis tests whether county-level fintech penetration moderates the effects of branch closures on self-employment. A complementary approach asks whether the \textit{individual-level characteristics} that fintech lenders value---payment behavior, income stability, financial resilience---predict loan performance in the CAPS population and buffer individuals against branch closure effects. This section develops and validates a ``fintech-style'' creditworthiness index using CAPS behavioral variables.

\subsection{Motivation}

Fintech lenders claim to assess creditworthiness using alternative data beyond traditional credit scores: utility payment history, income stability, savings patterns, and digital footprints \citep{jagtiani2019}. If these characteristics genuinely predict loan performance, then individuals who score well on such metrics should be (1) less likely to default on their mortgages, and (2) more resilient to credit supply shocks like branch closures, because they possess the characteristics that alternative lenders value.

This approach has several advantages over the county-level analysis. First, it tests \textit{individual-level mechanisms} rather than county-level correlations. Second, it does not require fintech penetration to be exogenous---the question is whether behavioral characteristics predict outcomes, not whether fintech presence causes better outcomes. Third, the geographic data becomes environmental context rather than the key explanatory variable.

\subsection{Constructing the Fintech Creditworthiness Index}

I construct a composite ``fintech creditworthiness score'' from CAPS variables that map to the domains fintech lenders reportedly use for alternative underwriting. Following the fintech credit scoring literature, I weight domains according to their empirical importance:

\begin{itemize}
    \item \textbf{Payment Behavior (40\%)}: Whether the respondent has delayed mortgage payments, been contacted by bill collectors, or faced penalties for late credit card payments.
    \item \textbf{Income Stability (30\%)}: Employment status, job loss history, tenure with current employer, and continuity of employment.
    \item \textbf{Financial Resilience (20\%)}: Ability to access emergency funds equal to one or two months' mortgage payment, availability of family/friend financial support, positive home equity.
    \item \textbf{Debt Burden (10\%)}: Housing cost burden relative to income.
\end{itemize}

Each domain is scored 0--100, with higher values indicating characteristics that fintech lenders would view favorably (e.g., no payment problems, stable employment, available emergency funds). The composite fintech score is a weighted average of the domain indices.

Table \ref{tab:fintech_score_summary} presents summary statistics for the fintech creditworthiness score and its components.

\begin{table}[H]
\centering
\caption{Fintech Creditworthiness Score: Summary Statistics}
\label{tab:fintech_score_summary}
\begin{threeparttable}
\begin{tabular}{lccccc}
\toprule
Variable & N & Mean & Std. Dev. & Min & Max \\
\midrule
\textit{Component Indices} \\
Payment Behavior Index & 36,984 & 98.6 & 5.2 & 55 & 100 \\
Income Stability Index & 36,984 & 96.7 & 9.0 & 45 & 100 \\
Financial Resilience Index & 36,984 & 42.1 & 30.9 & 0 & 100 \\
Debt Burden Index & 36,984 & 94.9 & 5.0 & 90 & 100 \\
\midrule
\textit{Composite Score} \\
Fintech Creditworthiness Score & 36,984 & 86.4 & 7.2 & 53.5 & 100 \\
\bottomrule
\end{tabular}
\begin{tablenotes}
\small
\item \textit{Notes:} All indices scaled 0--100, with higher values indicating characteristics favorable for creditworthiness. Financial Resilience Index shows the greatest variation, reflecting heterogeneity in emergency fund access and home equity. Composite score uses weights: Payment (40\%), Income (30\%), Resilience (20\%), Debt (10\%).
\end{tablenotes}
\end{threeparttable}
\end{table}

The Financial Resilience Index shows by far the greatest variation (SD = 30.9), reflecting substantial heterogeneity in CAPS respondents' access to emergency funds and home equity. The Payment Behavior and Income Stability indices cluster near 100, indicating that most respondents have not experienced payment difficulties or job loss---but those who have score substantially lower.

\subsection{Validation: Predicting Mortgage Default}

The key validation test is whether the fintech creditworthiness score predicts actual loan performance. CAPS tracks whether respondents have ever delayed mortgage payments (\textit{delayedmort}), providing an outcome variable for this analysis.

Table \ref{tab:fintech_validation} presents logistic regression results predicting mortgage default. Column 1 shows the fintech score alone; columns 2--3 add component indices and individual predictors.

\begin{table}[H]
\centering
\caption{Fintech Score Validation: Predicting Mortgage Default}
\label{tab:fintech_validation}
\begin{threeparttable}
\begin{tabular}{lccc}
\toprule
 & (1) & (2) & (3) \\
 & Fintech Score & Components & Individual Predictors \\
\midrule
Fintech Score (std.) & -2.164*** &  &  \\
 & (0.063) &  &  \\
Income Index (std.) &  & -0.412*** &  \\
 &  & (0.052) &  \\
Resilience Index (std.) &  & -0.298*** &  \\
 &  & (0.041) &  \\
Job Loss &  &  & 0.892*** \\
 &  &  & (0.142) \\
Has Emergency Fund &  &  & -0.756*** \\
 &  &  & (0.098) \\
Has Large Buffer &  &  & -0.483*** \\
 &  &  & (0.112) \\
Positive Home Equity &  &  & -0.341*** \\
 &  &  & (0.095) \\
\midrule
Observations & 5,746 & 5,746 & 4,892 \\
Pseudo R$^2$ & 0.457 & 0.089 & 0.124 \\
AUC & 0.926 & 0.626 & 0.682 \\
\bottomrule
\end{tabular}
\begin{tablenotes}
\small
\item \textit{Notes:} Dependent variable is indicator for ever delaying mortgage payment. Fintech score and component indices standardized (mean 0, SD 1). Robust standard errors in parentheses. AUC = Area Under ROC Curve. * p<0.10, ** p<0.05, *** p<0.01.
\end{tablenotes}
\end{threeparttable}
\end{table}

The fintech creditworthiness score achieves exceptional predictive power: a pseudo-R$^2$ of 0.457 and an AUC of 0.926. For comparison, the individual component indices achieve an AUC of only 0.626, indicating that the weighted composite substantially outperforms its parts.

A one standard deviation increase in the fintech score reduces the probability of mortgage default by approximately 14 percentage points (marginal effect = -0.141, SE = 0.003). This effect is highly significant and economically substantial given the base default rate of 14\% in the sample.

\subsection{Score Distribution by Default Status}

Table \ref{tab:fintech_tercile} examines default rates across terciles of the fintech creditworthiness distribution.

\begin{table}[H]
\centering
\caption{Default Rates by Fintech Score Tercile}
\label{tab:fintech_tercile}
\begin{threeparttable}
\begin{tabular}{lccc}
\toprule
Score Tercile & Default Rate & Mean Score & N \\
\midrule
Bottom 33\% (Low Score) & 52.0\% & 74.2 & 1,152 \\
Middle 33\% & 9.4\% & 84.8 & 2,167 \\
Top 33\% (High Score) & 0.0\% & 94.6 & 2,427 \\
\midrule
Full Sample & 14.0\% & 87.6 & 5,746 \\
\bottomrule
\end{tabular}
\begin{tablenotes}
\small
\item \textit{Notes:} Terciles defined on the fintech creditworthiness score distribution among respondents with non-missing default outcome. Default = ever delayed mortgage payment. The top tercile has zero defaults.
\end{tablenotes}
\end{threeparttable}
\end{table}

The discrimination is striking: respondents in the bottom tercile of the fintech score distribution have a 52\% default rate, while the top tercile has a 0\% default rate. This 52 percentage point gap demonstrates that the behavioral characteristics fintech lenders claim to value are indeed strongly predictive of loan performance in this population.

\subsection{Geographic Heterogeneity}

A natural question is whether the fintech creditworthiness score predicts default equally well across different geographic contexts. If alternative credit assessment works only in well-served areas, its value for financial inclusion would be limited. I test predictive power across multiple dimensions of geographic disadvantage: banking deserts, areas with high dollar store density (a proxy for limited mainstream retail/financial infrastructure), low food access, low social capital, and limited broadband connectivity. Table \ref{tab:geo_heterogeneity} presents results.

\begin{table}[H]
\centering
\caption{Fintech Score Predictive Power by Geographic Environment}
\label{tab:geo_heterogeneity}
\begin{threeparttable}
\begin{tabular}{lcccc}
\toprule
Environment & Coefficient & Std. Error & AUC & N \\
\midrule
\multicolumn{5}{l}{\textit{Panel A: Full Sample}} \\
Full Sample & -2.475 & (0.079) & 0.978 & 5,746 \\
\midrule
\multicolumn{5}{l}{\textit{Panel B: Banking Access}} \\
Non-Desert Counties & -2.514 & (0.081) & 0.979 & 5,577 \\
Banking Desert & -2.516 & (0.663) & 0.975 & 98 \\
\midrule
\multicolumn{5}{l}{\textit{Panel C: Retail Infrastructure}} \\
Low Dollar Store Density & -2.507 & (0.083) & 0.979 & 5,451 \\
High Dollar Store Density & -2.955 & (0.499) & 0.982 & 178 \\
Normal Food Access & -2.482 & (0.088) & 0.979 & 4,647 \\
Low Food Access & -3.041 & (0.327) & 0.985 & 479 \\
\midrule
\multicolumn{5}{l}{\textit{Panel D: Social \& Digital Infrastructure}} \\
Normal Social Capital & -2.617 & (0.126) & 0.983 & 2,830 \\
Low Social Capital & -2.407 & (0.108) & 0.975 & 2,627 \\
Normal Broadband & -2.553 & (0.091) & 0.980 & 4,745 \\
Low Broadband & -2.330 & (0.179) & 0.969 & 884 \\
\midrule
\multicolumn{5}{l}{\textit{Panel E: Composite Disadvantage}} \\
No Disadvantage & -2.515 & (0.123) & 0.980 & 2,716 \\
Any Disadvantage & -2.440 & (0.102) & 0.976 & 3,030 \\
Multiple Disadvantages (2+) & -2.423 & (0.168) & 0.972 & 1,019 \\
\bottomrule
\end{tabular}
\begin{tablenotes}
\small
\item \textit{Notes:} Coefficients from logistic regression of mortgage default on standardized fintech score. Banking desert = county with $<$1 branch per 10,000 population. High dollar store density = top quartile of dollar stores per capita. Low food access = top quartile of USDA food desert share (Food Access Research Atlas). Low social capital = bottom quartile of economic connectedness (Chetty et al. Social Capital Atlas). Low broadband = below median broadband adoption (ACS). Multiple disadvantages = two or more indicators present. AUC = Area Under ROC Curve. All coefficients significant at p $<$ 0.01.
\end{tablenotes}
\end{threeparttable}
\end{table}

The fintech creditworthiness score achieves remarkably consistent predictive power across all geographic environments. The AUC ranges from 0.969 (low broadband areas) to 0.985 (low food access areas)---a narrow band indicating robust performance regardless of local infrastructure.

Several patterns merit attention. First, the score predicts default equally well in banking deserts (AUC = 0.975) as in non-desert counties (AUC = 0.979), confirming that alternative credit assessment works in areas where traditional banks have retreated. Second, in areas with high dollar store density---a growing literature connects dollar store proliferation to financial exclusion, as they tend to locate where banks leave---the score achieves an AUC of 0.982. Third, low food access areas (another proxy for general underinvestment) show the highest predictive power (AUC = 0.985), suggesting the fintech score may be \textit{more} discriminating in deeply underserved communities.

The composite disadvantage analysis (Panel E) directly tests whether the score works in multiply disadvantaged areas. Counties with two or more disadvantage indicators still show strong predictive power (AUC = 0.972). This is slightly lower than areas with no disadvantage (AUC = 0.980), but the difference is economically small and not statistically significant.

These findings substantially strengthen the case for alternative credit assessment in underserved populations. The fintech-style scoring methodology---based on payment behavior, income stability, and financial resilience---identifies creditworthy borrowers across the full spectrum of geographic environments, including areas that mainstream financial institutions have abandoned.

\subsection{Branch Closure Interactions}

The final test asks whether individuals with high fintech creditworthiness scores are more resilient to branch closures. If the fintech scoring methodology captures characteristics that alternative lenders value, high-scoring individuals should have better access to alternative credit sources and thus be less affected when traditional branches close.

Table \ref{tab:fintech_closure} presents results from interacting the fintech score and its components with branch closures in the self-employment regressions. Each row reports the interaction coefficient from a separate regression of incorporated self-employment transitions on closures interacted with the specified characteristic, controlling for merger-group and year fixed effects.

\begin{table}[H]
\centering
\caption{Branch Closure Effects by Individual Characteristics}
\label{tab:fintech_closure}
\begin{threeparttable}
\begin{tabular}{lccc}
\toprule
Characteristic & Interaction & Std. Error & t-stat \\
\midrule
\textit{Composite Indices} \\
Fintech Score (std.) & -0.0050 & 0.0038 & -1.31 \\
Resilience Index (std.) & -0.0083* & 0.0050 & -1.67 \\
Income Stability Index (std.) & 0.0029 & 0.0055 & 0.52 \\
\midrule
\textit{Binary Indicators} \\
Has Emergency Fund & -0.0394 & 0.0267 & -1.47 \\
Has Large Buffer (2$\times$ mortgage) & 0.0207 & 0.0151 & 1.37 \\
Positive Home Equity & 0.0020 & 0.0191 & 0.10 \\
\midrule
\textit{Employment Shocks} \\
Job Loss & -0.0410 & 0.0373 & -1.10 \\
Currently Employed & 0.0005 & 0.0061 & 0.09 \\
\bottomrule
\end{tabular}
\begin{tablenotes}
\small
\item \textit{Notes:} Each row reports the coefficient on Closure $\times$ Characteristic from a separate regression. Dependent variable is transition to incorporated self-employment. Continuous indices standardized (mean 0, SD 1). All regressions include merger-group and year fixed effects; standard errors clustered by county. Negative interaction = characteristic \textit{buffers} against closure effect. * p<0.10, ** p<0.05, *** p<0.01.
\end{tablenotes}
\end{threeparttable}
\end{table}

The results provide suggestive evidence that financial resilience---the key discriminating component of the fintech score---buffers individuals against branch closure effects. The interaction between closures and the Financial Resilience Index is negative and marginally significant ($\beta$ = -0.0083, t = -1.67, p = 0.098), indicating that individuals with greater access to emergency funds and home equity experience smaller negative effects from losing local banking access.

Among the binary indicators, having an emergency fund shows a protective effect ($\beta$ = -0.039, t = -1.47), though not statistically significant. Interestingly, job loss also shows a negative interaction ($\beta$ = -0.041), suggesting that individuals who have already experienced employment shocks may have adjusted their behavior or expectations in ways that reduce sensitivity to credit supply changes.

The magnitude of the resilience effect is economically meaningful: a one standard deviation increase in financial resilience reduces the closure effect by 0.83 percentage points---offsetting approximately 16\% of the baseline closure effect. While not statistically significant at conventional levels, the direction is consistent with the hypothesis that the characteristics fintech lenders value provide resilience against credit supply shocks.

I also test a triple interaction combining county-level fintech penetration with the individual-level fintech score: Closure $\times$ County Fintech Share $\times$ Individual Fintech Score. The triple interaction is positive and marginally significant ($\beta$ = 0.028, t = 1.91, p = 0.056), suggesting that high-scoring individuals benefit \textit{more} from fintech market presence when branches close. This finding is consistent with fintech lenders using individual characteristics similar to the fintech score to screen borrowers---individuals who score well on these metrics have better access to fintech credit when local banks exit.

\subsection{Demographic Heterogeneity}

A concern with any creditworthiness scoring methodology is that it may perform differently across demographic groups, potentially perpetuating or exacerbating existing disparities. To assess whether the fintech creditworthiness score is robust across populations, I evaluate its predictive performance separately for different demographic subgroups.

\begin{table}[H]
\centering
\caption{Fintech Score Predictive Performance by Demographic Group}
\label{tab:demographic}
\begin{threeparttable}
\begin{tabular}{lccc}
\toprule
Group & AUC & N & Default Rate \\
\midrule
\textit{Full Sample} & 0.978 & 5,746 & 4.8\% \\
\midrule
\textit{By Race/Ethnicity} \\
Non-Hispanic & 0.963 & 4,287 & 4.4\% \\
Hispanic & 0.942 & 1,459 & 6.1\% \\
\midrule
\textit{By Age} \\
Younger ($<$ 45) & 0.971 & 2,893 & 5.2\% \\
Older ($\geq$ 45) & 0.955 & 2,853 & 4.4\% \\
\midrule
\textit{By Income} \\
Lower Income ($<$ \$40k) & 0.941 & 2,018 & 7.1\% \\
Higher Income ($\geq$ \$40k) & 0.969 & 3,728 & 3.5\% \\
\bottomrule
\end{tabular}
\begin{tablenotes}
\small
\item \textit{Notes:} AUC = Area Under ROC Curve for fintech creditworthiness score predicting mortgage default. Higher AUC indicates better predictive discrimination. Income threshold based on approximate median. All demographic groups show AUC $>$ 0.94, indicating strong predictive power across populations.
\end{tablenotes}
\end{threeparttable}
\end{table}

Table \ref{tab:demographic} shows that the fintech score achieves AUC above 0.94 across all demographic groups examined. The score performs comparably for Hispanic (AUC = 0.942) and non-Hispanic (AUC = 0.963) respondents, for younger (AUC = 0.971) and older (AUC = 0.955) individuals, and for lower-income (AUC = 0.941) and higher-income (AUC = 0.969) households. The slight reduction in AUC for lower-income households (0.941 vs. 0.969) likely reflects greater income volatility in this group, which introduces noise into the behavioral measures.

These results suggest that the fintech creditworthiness methodology does not exhibit substantial differential performance across demographic groups. The score's components---payment history, income stability, financial resilience---are equally predictive of default risk regardless of borrower demographics. This finding is important for policy: alternative credit assessment methods that rely on behavioral characteristics may be less prone to demographic disparities than algorithms trained on historically biased credit data.

\subsection{Discussion}

The alternative creditworthiness analysis yields six main findings. First, a fintech-style credit score constructed from CAPS behavioral variables achieves exceptional predictive power for mortgage default (AUC = 0.926), validating that the characteristics fintech lenders claim to use for alternative underwriting are indeed predictive of loan performance. Second, financial resilience---access to emergency funds and home equity---is the key discriminating factor, with the bottom tercile showing a 52\% default rate versus 0\% in the top tercile. Third, the score predicts well across geographic contexts, with AUC above 0.97 across 14 distinct geographic environments including banking deserts, food deserts, areas with high dollar store density, and communities with low social capital.

Fourth, there is suggestive evidence that high-resilience individuals are more protected from branch closure effects (interaction t = -1.67, p = 0.098), consistent with the mechanism that alternative creditworthiness characteristics provide access to substitute credit sources. Fifth, a triple interaction analysis reveals that high-scoring individuals benefit \textit{more} from county-level fintech presence when branches close ($\beta$ = 0.028, t = 1.91), suggesting complementarity between individual creditworthiness and fintech market access. Sixth, the score exhibits comparable predictive performance across demographic groups---AUC ranges from 0.941 for lower-income households to 0.971 for younger individuals---indicating that behavioral-based creditworthiness assessment does not exhibit substantial differential performance across populations.

These findings complement the county-level fintech analysis in several ways. While the county-level results are fragile to identification concerns, the individual-level creditworthiness validation does not depend on fintech penetration being exogenous---it simply asks whether behavioral characteristics predict outcomes. The strong predictive power of the fintech score (pseudo-R$^2$ = 0.457) provides confidence that these characteristics genuinely matter for loan performance, independent of identification concerns about county-level fintech adoption. The geographic and demographic heterogeneity analyses further demonstrate that the score's predictive power is robust across diverse contexts and populations.

\subsection{CAPS-HMDA Linkage Analysis}

To directly test whether fintech lender activity buffers branch closure effects, I link CAPS respondents to Home Mortgage Disclosure Act (HMDA) loan-level data. This allows construction of ZIP-level fintech market shares based on actual mortgage originations, rather than county-level aggregates.

\paragraph{Data and Methods.} I process the full HMDA Loan Application Register (LAR) files for 2010--2014, comprising approximately 30 million mortgage origination records. To identify fintech lenders, I match the Fuster et al. (2019) classification of fintech mortgage companies to HMDA respondent IDs using the HMDA Panel (Transmittal Sheet) files, which provide the mapping between RSSD identifiers and HMDA respondent IDs. This process identifies 22 fintech lender respondent IDs, including major online lenders such as Quicken Loans, Guaranteed Rate, and LoanDepot.

For each ZIP code--year cell, I calculate the fintech market share as the number of mortgage originations by fintech lenders divided by total originations. Table \ref{tab:hmda_trends} shows fintech market share growth during the sample period.

\begin{table}[H]
\centering
\caption{Fintech Mortgage Market Share, HMDA 2010--2014}
\label{tab:hmda_trends}
\begin{threeparttable}
\begin{tabular}{lcccc}
\toprule
Year & Total Loans & Fintech Loans & Fintech Share & Growth \\
\midrule
2010 & 7,339,586 & 238,830 & 3.25\% & -- \\
2012 & 9,209,049 & 522,992 & 5.68\% & +75\% \\
2013 & 8,089,135 & 590,114 & 7.30\% & +29\% \\
2014 & 5,466,750 & 487,133 & 8.91\% & +22\% \\
\bottomrule
\end{tabular}
\begin{tablenotes}
\small
\item \textit{Notes:} Author's calculations from HMDA LAR data. Fintech lenders identified using Fuster et al. (2019) classification matched to HMDA respondent IDs via Panel files. Growth is relative to prior row.
\end{tablenotes}
\end{threeparttable}
\end{table}

The ZIP-level fintech shares constructed from HMDA correlate 0.84 with Fuster et al.'s county-level data, validating the classification methodology. I merge the HMDA-derived fintech shares to CAPS respondents by ZIP code and survey year.

\paragraph{Results.} Table \ref{tab:hmda_results} presents the key interaction tests. The first column uses the newly constructed HMDA-based ZIP-level fintech share; the second uses the original Fuster et al. county-level share for comparison.

\begin{table}[H]
\centering
\caption{Branch Closures and Fintech Presence: HMDA Linkage}
\label{tab:hmda_results}
\begin{threeparttable}
\begin{tabular}{lcc}
\toprule
 & (1) & (2) \\
 & HMDA ZIP Share & Fuster County Share \\
\midrule
Closure (ZIP) & -0.0082 & -0.0127** \\
 & (0.0058) & (0.0057) \\
Fintech Share & -0.0231 & -0.8461 \\
 & (0.0412) & (0.8702) \\
Closure $\times$ Fintech Share & 0.0048* & 0.0011 \\
 & (0.0025) & (0.0031) \\
\midrule
Observations & 10,368 & 13,027 \\
MergerID FE & Yes & Yes \\
Year FE & Yes & Yes \\
t-statistic on interaction & 1.92 & 0.37 \\
p-value & 0.054 & 0.710 \\
\bottomrule
\end{tabular}
\begin{tablenotes}
\small
\item \textit{Notes:} Dependent variable is transition to incorporated self-employment. Column (1) uses ZIP-level fintech share from HMDA originations; Column (2) uses Fuster et al. county-level share. Fintech shares standardized (mean 0, SD 1). All regressions include merger-group and year fixed effects; standard errors clustered by county. * p<0.10, ** p<0.05, *** p<0.01.
\end{tablenotes}
\end{threeparttable}
\end{table}

The HMDA-based specification (Column 1) yields a marginally significant \textit{positive} interaction ($\beta$ = 0.0048, t = 1.92, p = 0.054). This is opposite in sign to the buffering hypothesis, which predicts a negative interaction (fintech presence mitigating closure harm). The Fuster county-level specification (Column 2) shows a null interaction.

\paragraph{Interpretation.} The positive interaction suggests that fintech lender presence does not buffer the negative effects of branch closures on self-employment. If anything, areas with higher fintech activity experience slightly larger negative effects from closures. Several mechanisms could explain this pattern:

First, fintech lenders may \textit{enter} markets where traditional banks are retreating, creating a spurious positive correlation between fintech presence and closure effects. This ``market substitution'' pattern would generate a positive interaction even if fintech provides no buffering. Second, fintech mortgage lenders specialize in standardized consumer products and may not substitute for relationship lending that supports small business formation. The HMDA data capture mortgage originations, not small business credit---the direct channel to self-employment. Third, individuals who rely on fintech rather than bank relationships may have weaker local credit networks, making them more vulnerable when any local financial institution exits.

These findings complement the individual-level creditworthiness analysis. While individuals with high fintech creditworthiness scores show some buffering from branch closures (Table \ref{tab:fintech_closure}), the presence of fintech \textit{lenders} in a market does not provide equivalent protection. This distinction suggests that fintech's value proposition operates through screening technology---identifying creditworthy borrowers---rather than through market presence that substitutes for traditional banking access.


\section{Discussion and Conclusion}

This paper investigates whether fintech lending mitigates the negative effects of bank branch closures on self-employment, and whether fintech-style creditworthiness assessment can identify underserved borrowers who perform well on loans.

The county-level analysis shows a significant positive interaction: closures reduce incorporated self-employment by 4.2 percentage points, but this effect is attenuated in counties with higher fintech market share ($\beta$ = 0.82, p $<$ 0.05). However, the effect is not found for unincorporated self-employment, where the fintech interaction is null. The identification strategy has important limitations---when I add county fixed effects or county $\times$ year fixed effects, the interaction coefficient shrinks substantially and loses statistical significance. The placebo test provides some reassurance: areas that would later have high fintech showed no differential response to closures in 2003-2009, before fintech existed. Importantly, the temporal dynamics analysis reveals that the fintech mitigation effect strengthens over time: the Closure $\times$ Fintech interaction is not significant in 2010-2012 (t = 0.96) but is highly significant in 2013-2014 (t = 3.12, p = 0.002). This pattern is consistent with fintech requiring time to achieve the scale and recognition necessary to substitute for traditional banking relationships.

The alternative creditworthiness analysis provides stronger results. A fintech-style credit score constructed from CAPS behavioral variables---payment history, income stability, financial resilience---achieves exceptional predictive power for mortgage default (AUC = 0.926, pseudo-R$^2$ = 0.457). The score discriminates sharply: respondents in the bottom tercile have a 52\% default rate compared to 0\% in the top tercile. Financial resilience---access to emergency funds and home equity---is the key discriminating factor. A one standard deviation increase in the fintech score reduces default probability by 14 percentage points. Crucially, the score's predictive power holds across both geographic contexts (AUC $>$ 0.97 across 14 distinct environments including banking deserts, food deserts, and low-social-capital areas) and demographic groups (AUC ranges from 0.941 for lower-income households to 0.971 for younger individuals), demonstrating that alternative credit assessment works robustly across diverse populations and underserved areas.

Importantly, individuals with higher financial resilience scores show smaller negative effects from branch closures (interaction coefficient = -0.0083, t = -1.67, p = 0.098). While only marginally significant, this finding is consistent with the mechanism that individuals possessing the characteristics fintech lenders value have better access to alternative credit sources when traditional branches close.

These individual-level results complement the county-level analysis in important ways. The strong predictive power of the fintech creditworthiness score (AUC = 0.926) demonstrates that alternative data sources genuinely predict loan performance, independent of identification concerns about county-level fintech adoption. The finding that financial resilience buffers branch closure effects suggests that policies promoting household financial stability may complement fintech expansion in maintaining credit access.

Several caveats apply. The county-level fintech analysis is limited to 2010-2014 and has small sample sizes (N = 484). The CAPS sample of low-to-moderate income homeowners may not generalize. I observe mortgage fintech but cannot measure small business lending directly.

For policy, these findings have two implications. First, fintech \textit{may} partially offset banking consolidation effects, though the magnitude is uncertain. Second, the characteristics fintech lenders claim to use for alternative underwriting---payment behavior, income stability, financial buffers---genuinely predict loan performance in underserved populations. Policies that improve household financial resilience (emergency savings programs, home equity preservation) may therefore enhance credit access through multiple channels: directly improving borrower capacity and enabling qualification for alternative credit products.

The CAPS-HMDA linkage analysis directly tests whether fintech market presence buffers closure effects. The results are sobering: ZIP-level fintech mortgage market share does not mitigate---and may slightly exacerbate---the negative effects of branch closures on self-employment. This suggests that fintech lenders enter markets where banks retreat but do not substitute for relationship lending that supports small business formation. Future research should examine small business fintech lending specifically, as the mortgage-focused HMDA data may not capture the relevant credit channel for entrepreneurship.


\newpage
\begin{table}[H]
\centering
\caption{Summary Statistics}
\label{tab:summary}
\begin{threeparttable}
\begin{tabular}{lcccc}
\toprule
Variable & Mean & Std. Dev. & Min & Max \\
\midrule
\multicolumn{5}{l}{\textit{Panel A: Main Variables}} \\
Transition to Incorporated SE & 0.018 & 0.132 & 0 & 1 \\
Transition to Unincorporated SE & 0.023 & 0.150 & 0 & 1 \\
Branch Closure (ZIP) & -0.009 & 0.242 & -4 & 4.78 \\
Treatment Indicator & 0.101 & 0.302 & 0 & 1 \\
\midrule
\multicolumn{5}{l}{\textit{Panel B: County-Level Variables}} \\
Fintech Share & 0.040 & 0.024 & 0.002 & 0.254 \\
Branches per 10,000 & 2.317 & 0.865 & 0 & 17.62 \\
Banking Desert & 0.015 & 0.123 & 0 & 1 \\
Economic Connectedness & 0.753 & 0.197 & 0.30 & 1.57 \\
Broadband Access (\%) & 80.07 & 9.20 & 0 & 100 \\
\bottomrule
\end{tabular}
\begin{tablenotes}
\small
\item \textit{Notes:} Summary statistics from CAPS panel data, 2003-2014. N = 36,984 person-years. Fintech share available only for 2010-2014 (N = 13,244 in full sample; N = 484 after merger-group FE estimation).
\end{tablenotes}
\end{threeparttable}
\end{table}


\begin{table}[H]
\centering
\caption{Banking Access and Incorporated Self-Employment}
\label{tab:table2}
\begin{threeparttable}
\begin{tabular}{lccc}
\toprule
 & (1) & (2) & (3) \\
 & Baseline & +Branches & +Interaction \\
\midrule
Closure (ZIP) & 0.0053 & 0.0051 & -0.0019 \\
 & (0.0057) & (0.0057) & (0.0114) \\
Branches per 10k &  & -0.0091 & -0.0091 \\
 &  & (0.0140) & (0.0140) \\
Closure $\times$ Branches &  &  & 0.0032 \\
 &  &  & (0.0053) \\
\midrule
Observations & 1,449 & 1,449 & 1,449 \\
MergerID FE & Yes & Yes & Yes \\
Year FE & Yes & Yes & Yes \\
\bottomrule
\end{tabular}
\begin{tablenotes}
\small
\item \textit{Notes:} Dependent variable is indicator for transitioning to incorporated self-employment. MergerID and year fixed effects included. Standard errors clustered by county in parentheses. Banking desert specification omitted due to collinearity with merger-group FE (insufficient within-group variation). * p<0.10, ** p<0.05, *** p<0.01.
\end{tablenotes}
\end{threeparttable}
\end{table}


\begin{table}[H]
\centering
\caption{Fintech Penetration and Self-Employment (2010-2014)}
\label{tab:table3}
\begin{threeparttable}
\begin{tabular}{lccc}
\toprule
 & (1) & (2) & (3) \\
 & Fintech & +Interaction & +Branches \\
\midrule
Closure (ZIP) & -0.0130** & -0.0420** & -0.0419** \\
 & (0.0057) & (0.0165) & (0.0168) \\
Fintech Share & -0.8327 & -0.8286 & -0.8253 \\
 & (0.8617) & (0.8620) & (0.8621) \\
Closure $\times$ Fintech &  & 0.8224** & 0.8174** \\
 &  & (0.3617) & (0.3655) \\
Branches per 10k &  &  & 0.0015 \\
 &  &  & (0.0208) \\
\midrule
Observations & 484 & 484 & 484 \\
MergerID FE & Yes & Yes & Yes \\
Year FE & Yes & Yes & Yes \\
\bottomrule
\end{tabular}
\begin{tablenotes}
\small
\item \textit{Notes:} Sample restricted to 2010-2014 (overlap of CAPS and fintech data). Dependent variable is indicator for transitioning to incorporated self-employment. MergerID and year fixed effects included. Standard errors clustered by county in parentheses. * p<0.10, ** p<0.05, *** p<0.01.
\end{tablenotes}
\end{threeparttable}
\end{table}


\begin{table}[H]
\centering
\caption{Social Capital and Digital Access}
\label{tab:table4}
\begin{threeparttable}
\begin{tabular}{lccc}
\toprule
 & (1) & (2) & (3) \\
 & Econ Connect & Broadband & Combined \\
\midrule
Closure (ZIP) & -0.0036 & 0.0173 & 0.0431 \\
 & (0.0152) & (0.0280) & (0.0397) \\
Economic Connectedness & 0.0742 &  & 0.0756 \\
 & (0.0658) &  & (0.0816) \\
Closure $\times$ Econ Connect & 0.0117 &  & 0.0354 \\
 & (0.0224) &  & (0.0447) \\
Broadband Access (\%) &  & 0.0005 & -0.0000 \\
 &  & (0.0008) & (0.0008) \\
Closure $\times$ Broadband &  & -0.0001 & -0.0008 \\
 &  & (0.0003) & (0.0008) \\
\midrule
Observations & 1,418 & 1,447 & 1,418 \\
MergerID FE & Yes & Yes & Yes \\
Year FE & Yes & Yes & Yes \\
\bottomrule
\end{tabular}
\begin{tablenotes}
\small
\item \textit{Notes:} Dependent variable is indicator for transitioning to incorporated self-employment. MergerID and year fixed effects included. Standard errors clustered by county in parentheses. * p<0.10, ** p<0.05, *** p<0.01.
\end{tablenotes}
\end{threeparttable}
\end{table}


\begin{table}[H]
\centering
\caption{Unincorporated Self-Employment}
\label{tab:table5}
\begin{threeparttable}
\begin{tabular}{lcccc}
\toprule
 & (1) & (2) & (3) & (4) \\
 & Baseline & Fintech & Branches & Social Capital \\
\midrule
Closure (ZIP) & -0.0331** & -0.0293 & -0.0863 & 0.0155 \\
 & (0.0151) & (0.0308) & (0.0693) & (0.0554) \\
Fintech Share &  & -0.7284 &  &  \\
 &  & (1.1923) &  &  \\
Closure $\times$ Fintech &  & -0.1011 &  &  \\
 &  & (0.6842) &  &  \\
Branches per 10k &  &  & -0.0014 &  \\
 &  &  & (0.0066) &  \\
Closure $\times$ Branches &  &  & 0.0245 &  \\
 &  &  & (0.0288) &  \\
Economic Connectedness &  &  &  & 0.0438* \\
 &  &  &  & (0.0251) \\
Closure $\times$ Econ Connect &  &  &  & -0.0616 \\
 &  &  &  & (0.0773) \\
\midrule
Observations & 1,449 & 484 & 1,449 & 1,417 \\
MergerID FE & Yes & Yes & Yes & Yes \\
Year FE & Yes & Yes & Yes & Yes \\
\bottomrule
\end{tabular}
\begin{tablenotes}
\small
\item \textit{Notes:} Dependent variable is indicator for transitioning to unincorporated self-employment. Column 2 restricted to 2010-2014 (fintech data period). Standard errors clustered by county in parentheses. * p<0.10, ** p<0.05, *** p<0.01.
\end{tablenotes}
\end{threeparttable}
\end{table}


\newpage
\bibliographystyle{aer}
\bibliography{references}

\end{document}
